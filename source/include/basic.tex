% Copyright (C) 2018 by latexstudio <http://www.latexstudio.net>
%
% This program is free software: you can redistribute it and/or modify
% it under the terms of the GNU General Public License as published by
% the Free Software Foundation, either version 3 of the License, or
% (at your option) any later version.
%
% This program is distributed in the hope that it will be useful,
% but WITHOUT ANY WARRANTY; without even the implied warranty of
% MERCHANTABILITY or FITNESS FOR A PARTICULAR PURPOSE.  See the
% GNU General Public License for more details.
%
% You should have received a copy of the GNU General Public License
% along with this program.  If not, see <http://www.gnu.org/licenses/>.
%

\section{背景知识与基本概念}
\label{sec:basic}

\faq{什么是 \TeX{}?}{what-is-tex}

\TeX{} 是由著名计算机科学家 Donald~E. Knuth(高德纳)发明的排版系统。他在 \book{The \TeX book} 一书
的前言中曾提到:“(\TeX{})旨在创造美丽的书籍,尤其是那些包含很多数学公式的书。”
\footnote{原文如下:“This is a handbook about \TeX{}, a new typesetting system intended for the
creation of beautiful books---and especially for books that contain a lot of mathematics.”}

1976 年,Knuth 出版鸿篇巨著《计算机程序设计艺术》(\book{The Art of Computer Programming})第二卷
的第二版,但当时所用的照排技术却令他非常失望。作为斯坦福大学计算机科学系的教授,Knuth 决定自己开发
一套高质量的排版系统。1978年,他开发出了 \TeX{} 的第一个版本;随后,又在 1982 年推出了 \TeX{} 的
第二个版本(\TeX 82),也就是人们今天所用 \TeX{} 的基础。Knuth 将 \TeX{} 的源代码无偿发布在公有领域
\footnote{\TeX{} 使用的许可证为 \href{https://www.ctan.org/license/knuth}{Knuth License}。},
这使得他人可以进一步完善这一系统,并增加新的功能。

在今天,\TeX{} 既可以指 Knuth 发明的这一套排版系统,也可以指相应的排版语言,有时候也指将其打包、
整理以方便用户使用的软件套装(发行版)。

\begin{reference}
  \item https://texfaq.org/FAQ-whatTeX
  \item \TeX{}, Wikipedia, The Free Encyclopedia, \url{https://en.wikipedia.org/wiki/TeX}
\end{reference}


\faq{\TeX{} 中常见术语的解释}{tex-terms}

\begin{description}
  % TODO
%  \item[引擎] \LaTeX{}/\TeX{} 解析引擎,其实就是一个编译器,它输入一个 |.tex| 文件作为输入,根据
%    源文件的内容送入解析引擎和渲染引擎进行处理,并将排版的成果——文档编译输出,\LaTeX{}/\TeX{} 的
%    解析引擎目前有pdflatex、xelatex、lualatex等,它们都可以输出pdf文档文件(部分解析器可以输出dvi
%    文件),用于在多平台进行分发甚至打印出版。
  \item[引擎] \LaTeX{}/\TeX{} 解析引擎,其实就是一个编译器,它输入一个 |.tex| 文件作为输入,根据
  源文件的内容送入解析引擎和渲染引擎进行处理,并将排版的成果——文档编译输出,\LaTeX{}/\TeX{} 的解
  析引擎目前有\pdfTeX{}、\XeLaTeX{}等。
  \item[格式] \TeX{} 是存在各种不同的封装格式的,比如原生的 \TeX{} 或者 \LaTeX{},我们所使用的
    \LaTeX{} 只是\TeX{} 封装格式的其中一种,是目前流行的封装规范。
  \item[发行版] \LaTeX/\TeX{}都包含了成千上万个宏包,甚至有可能我们需要安装新的宏包,除了手动安装
    外,最好的方式就是利用发行版的宏包管理器,所谓发行版就是把\LaTeX/\TeX{}的相关组件打包,形成一个
    独立完善的\LaTeX/\TeX{}系统,目前流行的发行版有MiKTeX、proTeXt 以及TeXLive。
  \item[编译命令] 是实际调用的、结合了引擎和格式的命令。如 |xelatex| 命令是结合\XeTeX{}引擎和
  \LaTeX{}格式的一个编译命令。
\end{description}


% \faq{不同的TeX封装格式的区别?}[tex-format]
% TODO: 该问题跟上下两个重复
%
% \textbf{原生\TeX{}}
%
% TeX本身是一个基于控制序列的排版系统,它指示TeX如何在页面上放置文本。例如,|\hskip|指在文档中在
% 文档中插入一定数量的水平空间,而|\font|是指给文档中的文字定义一种给定的字体。TeX是完全可编程的,
% 它使用一种集成的宏脚本语言,支持变量,范围,条件执行,控制流和函数定义等。
%
% \textbf{TeX宏包(\TeX{}格式)}
%
% TeX的一些控制序列直接使用是单调乏味的;它们主要作为更高层次的构建快,因此更易于用户使用。例如,
% 在基础TeX中没有办法能够制定一段文字应该排版为更大的字体,相反,你必须了解当前的字体和大小,然后
% 加载一种同样字体但更大字号的字体。幸运的是,TeX是可编程的,它可以通过写一个宏将这些复杂性都隐藏在
% 一个简单的,新的控制序列之后。例如,我们可以通过 |\larger{my text}| 将“my text”定义为比当前更大的
% 字体。一些使用者会写一些完全由自己定义的宏集,然后再一些文档中重复使用,但,常见的还是依赖于由专家
% 编写的TeX宏包——一些宏的集合。为了方便用户,这些宏包通常与基本的TeX引擎结合到一个独立的可执行的
% 文件中。
%
% \textbf{TeXLive、MacTeX、MikTeX、CTex}
%
% TeXLive 由类 UNIX 系统上的 teTeX 发展并取而代之,最终成为跨平台的 TeX 发行版。TeXLive 自 2011 年
% 起以年份作为发行版的版本号,保持了一年一更的频率。MacTeX 是 macOS(OS X)系统下的一个定制化的
% TeXLive 版本,与 TeXLive 同步更新。MikTeX 是主要用于 Windows 平台的一个稳定发展的 TeX 发行版,
% 目前已开发出跨平台版本。中国的 LaTeX 用户应该对“CTeX 套装”比较熟悉,它是一个经过本地化配置的
% MikTeX,目前已不推荐使用。


\faq{\TeX{} 有哪些格式?}{tex-format}

\TeX{} 是一种排版文件的计算机程序,它需要一个计算机文件,根据 \TeX{} 的规则进行准备,并将其转换成
一种可以在高质量打印机上打印的形式,比如激光打印机,可以打印出一份与高质量的书籍和期刊相媲美的打印
文档。不包含数学公式或表格的简单文档可以很容易就生成,事实上,所有人都必须直接输入文本(只是遵循
不同的符号规则)。输入数学公式时比较复杂的,但当考虑到产生一些数学公式的复杂性时,\TeX{} 是相对容易
使用的,它可以产生大量的数学符号。

\TeX{} 包括各种不同的“方言”,其中包括 \LaTeX{}。\PlainTeX{} 是 \TeX{} 中最基础的版本,也是其他
“方言”的基础。为了用 \TeX{} 生成文档,我们必须首先在计算机上创建一个合适的输入文件,我们将 \TeX{}
程序应用到输入文件中,然后再用打印机打印由 \TeX{} 程序生成的所谓的“DVI”文件。

\begin{description}
  \item[\PlainTeX] Knuth 设计了一个名叫 \PlainTeX{} 的基本格式,以与低层次的原始 \TeX{} 呼应。
    这种格式是用 \TeX{} 处理文本时相当基本的部分,以致于我们有时都分不清到底哪条指令是真正的处理
    程序 \TeX{} 的原始命令,哪条是 \PlainTeX{} 格式的。大多数声称只使用 \TeX{} 的人,实际上指的是
    只用 \PlainTeX{}。

    \PlainTeX{} 也是其它格式的基础,这进一步加深了很多人认为 \TeX{} 和 \PlainTeX{} 是同一事物的
    印象。\PlainTeX{} 的重点还只是在于如何排版的层次上,而不是从一位作者的观点出发。对它的深层功能
    的进一步发掘,需要相当丰富的编程技巧。因此它的应用就局限于高级排版和程序设计人员。

    有关 \PlainTeX{} 的相关信息可见:\url{http://www.ntg.nl/doc/wilkins/pllong.pdf}

  \item[\LaTeX] 有两个版本,分别是 \LaTeXe{} 和 \LaTeX2.09,前者是当前使用的版本,后者是在 1994 年
    首次发布的公开版本,但现已过时。因此,\LaTeX{} 与 \LaTeXe{} 实际上是同义词。

    Leslie Lamport 开发的 \LaTeX{} 是当今世界上最流行和使用最为广泛的 \TeX{} 格式。它构筑在 Plain
    \TeX{} 的基础之上,并加进了很多的功能以使得使用者可以更为方便的利用 \TeX{} 的强大功能。使用
    \LaTeX{} 基本上不需要使用者自己设计命令和宏等,因为它已经替你做好了。因此,即使使用者并不是很
    了解 \TeX{},也可以在短时间内生成高质量的文档。对于生成复杂的数学公式,\LaTeX{} 也表现得更为
    出色。

    \LaTeX{} 自从二十世纪八十年代初问世以来,也在不断的发展。最初的正式版本为 2.09,在经过几年的
    发展之后,许多新的功能,机制被引入到 \LaTeX{} 中。在享受这些新功能带来的便利的同时,它所伴随的
    副作用也开始显现,这就是不兼容性。标准的 \LaTeX2.09,引入了“新字体选择框架”(NFSS)的 \LaTeX{}、
    % TODO
    SLiTeX、AMSLaTeX 等等,相互之间并不兼容。这给使用者和维护者都带来很大的麻烦。

    为结束这种糟糕的状况,Frank Mittelbach 等人成立了 \LaTeX3 项目小组,目标是建立一个最优的、有效
    的、统一的、标准的命令集合,即得到 \LaTeX{} 的一个新版本 3。这是一个长期目标,向这个目标迈出
    第一步就是在 1994 年发布的 \LaTeXe{}。\LaTeXe{} 采用了 NFSS 作为标准,加入了很多新的功能,同时
    还兼容旧的 \LaTeX2.09。\LaTeXe{} 每 6 个月更新一次,修正发现的错误并加入一些新的功能。在
    \LaTeX3 最终完成之前,\LaTeXe{} 将是标准的 \LaTeX{} 版本。

  \item[\ConTeXt] \ConTeXt{} 是 \TeX{} 的一种格式,是 Hans Hagen 和荷兰 Pragma-ADE 公司设计的一种
    高端的文档制造工具,官方网站为 \href{http://wiki.contextgarden.net}{\ConTeXt{} Garden}。
    \ConTeXt{} 就像 \LaTeX{} 是基于 \TeX{} 排班系统的文档制作系统,如果说 \LaTeX{} 将作者从排版的
    细节中隔离出来,那么 \ConTeXt{} 就是采用一种互补的方法来处理结构化的界面来处理排版,包括对
    颜色、背景、超链接、演示文稿、图形文本集成和条件编译的广泛支持, 对于数学、化学等科技文档的支持
    同等优秀甚至更为方便,而且其为了更容易实现各种华丽排版效果。它可以让用户对格式进行广泛的控制,
    同时又可以在不需要学习 \TeX{} 宏语言的情况下轻松地创建新的布局和样式。 因此 \ConTeXt{} 的图形
    功能要远远强于 \TeX{} 和 \LaTeX{},可以制作非常漂亮的 PDF 文档,特别适合做幻灯片和一些非正式的
    文档。

    \ConTeXt{} 将 \METAFONT{}、\METAPOST{} 的超集以及一个强大的矢量图形系统整合起来。\MetaFun{}
    可以作为一个独立的系统来生成数据,但是它的优势在于用精确的图形元素来增强 \ConTeXt{} 文档。

    目前,\ConTeXt{} 主要分为两个版本,分别是 Mark II 和 Mark IV(MkII 和 MkIV)。前者已处于稳定
    状态,只进行代码维护;而后者仍在活跃开发中。MkII 默认使用 \pdfTeX{} 引擎;而 MkIV 由于采用了
    新的字体机制,仅支持 \LuaTeX{} 引擎。(\LuaTeX{} 的发展也是由于 \ConTeXt{} 驱动)。

    注:CTAN 不支持 \ConTeXt{} 的发布。潜在的用户可以去
    \href{http://wiki.contextgarden.net/Main_Page}{ConTeXt Garden} 了解当前发行版的详细信息。

  \item[Texinfo] TeXinfo 是一个使用同一个源文件生成在线信息和打印输出的文档系统,所以只需要编写一个
    文档源文件,当工作被修改时,只需要修改源文件即可。其源文件的后缀名为 |texi| 或 |texinfo|。

    TeXinfo 是一门宏语言,就像 \LaTeX{} 一样,但是它的表达能力略弱,它的表观与 \TeX{} 的其他宏语言
    相似,只不过它的宏要以 |@| 开头,而 \TeX{} 系统中用 |\| 开头。

    你可以在 GNU Emacs 中编写以及将 TeXinfo 文件转化成 info 文件,你也可以在 makeinfo 中将 TeXinfo
    文件转换成 info 文件,然后再 info 中阅读,所以也不是必须依赖于 Emacs。这个发行版包括一个 Perl
    脚本,|texi2html|,可以将 TeXinfo 文件转换成 HTML,这种语言比 \LaTeX{} 更适合 HTML,所以将
    \LaTeX{} 转换成 HTML 的痛苦就可以避免了。

    当然,你也可以用 \pdfTeX{} 将输入文件转换成 PDF 格式。makeinfo 可将 texinfo 文档转换成 HTML、
    DocBook、Emacs info、XML 和纯文本。\TeX{}(或者 |texi2dvi| 和 |texi2pdf|),因为 \TeX{} 加载了
    普通的 \TeX{} 宏,而并不是 TeXinfo,所以 TeXinfo 文档必须以 \verb|\input texinfo \verb| 开头来加载
    \pkg{texinfo} 包。

  % TODO: 不常用,可删去
  \item[Eplain] Eplain 扩展并延伸了 \PlainTeX{} 的定义,它并不像 \ConTeXt{}。
\end{description}


\faq{\LaTeX2.09 和 \LaTeXe{} 有什么区别?}{latex2.09-latex2e-diff}

后者是前者的改进。从文件内容上看,两者最显著的不同在于 \LaTeX2.09 使用 |\documentstyle| 命令定义
文档类以及所包含宏包,如:

\begin{texinlist}
\documentstyle[twoside,epsfig]{article}
\end{texinlist}

而 \LaTeX2e{} 使用 |\documentclass| 命令设置文档类型,用 |\usepackage| 命令调用宏包。


\faq{\TeX{}, \LaTeX{}, pdflatex, xelatex, xetex等的区别和关系,什么时候用什么编译器编译}
  {build-all-diff}
% TODO: 非常混乱……

LaTeX 其实是目前使用最广泛的 \TeX{} 格式。xeTeX 是一种引擎(编译器),pdfLaTeX (xeLaTeX) 是命令,
他们分别结合了 pdfTeX(xeTeX) 引擎和 \LaTeX{} 格式。对于刚开始接触的人,建议处理英文时直接使用
pdfLaTeX,处理非英文时使用 XeLaTeX(并且用utf-8编码源文件)


%\faq{文本文件编码解读}{encoding}


\faq{\LaTeX{} 的源文件有什么要求?}{latex-source-file}

\LaTeX{} 的源文件是 |*.tex| 文件,是指 latex 编译器处理输入文件的源码,latex 编译器会对输入文件进行
解析,构造解析树,进行渲染,然后输出处理后的文档,完成一次编译过程,由于 \LaTeX{} 解析器可能对中文
文件名处理存在兼容性问题,不建议将 \LaTeX{} 的源文件的文件名设置为中文。


\faq{连字符如何在 \TeX{} 起作用?}{hyphen}
% TODO: 不应该放在这里,给的例子也很奇怪

如果 \LaTeX{} 遇到了很长的英文单词,仅在单词之间的位置断行无法生成宽度匀称的行时,就要考虑从单词
中间断开。对于绝大部分单词,\LaTeX{} 能够找到合适的断词位置,在断开的行尾加上连字符 |-|。如果一些
单词没能自动断词,我们可以在单词内手动使用 |\-| 命令指定断词的位置,如:

\begin{texinlist}
I think this is: su\-per\-cal\-%
i\-frag\-i\-lis\-tic\-ex\-pi\-%
al\-i\-do\-cious.
\end{texinlist}


%\faq{Unicode 和 \TeX{}}{unicode-and-tex}


\faq{常见的 \TeX{} 文件扩展名与文件用途}{extensions}
% TODO: 见如下链接
%   https://tex.stackexchange.com/q/7770/136923
%   https://tex.stackexchange.com/q/53240/136923
%   https://github.com/wspr/latex-auxfiles

常见的用户文件的扩展名与其用户如下:
\begin{itemize}
  \item |.tex| 文件。源文件,需用户编写。
  \item |.sty| 宏包文件。宏包的名称就是去掉扩展名的文件名。
  \item |.cls| 文档类文件。同样地,文档类名称就是文件名
  \item |.bib| \BibTeX{} 参考文献数据库文件。
  \item |.bst| \BibTeX{} 用到的参考文献格式模板。
  \item |.log| 排版引擎生成的日志文件,供排查错误使用。
  \item |.aux| \LaTeX{} 生成的主辅助文件,记录交叉引用、目录、参考文献的引用等。
  \item |.toc| \LaTeX{} 生成的目录记录文件。
  \item |.lof| \LaTeX{} 生成的图片目录记录文件。
  \item |.lot| \LaTeX{} 生成的表格目录记录文件。
  \item |.bbl| \BibTeX{} 生成的参考文献记录文件。
  \item |.blg| \BibTeX{} 生成的日志文件。
  \item |.idx| \LaTeX{} 生成的供 \pkg{makeindex} 处理的索引记录文件。
  \item |.ind| \pkg{makeindex} 处理 |.idx| 生成的格式化索引记录文件。
  \item |.ilg| \pkg{makeindex} 生成的日志文件。
  \item |.out| \pkg{hyperref} 宏包生成的 PDF 书签记录文件。
\end{itemize}


\faq{什么是 DVI 文件?}{what-is-dvi}
% TODO: 部分内容重复

DVI(device independent)文件为 \TeX{} 电子排版系统的输出文件。七十年代末,Knuth 在看到其多卷巨著
《The Art of ComputerProgramming》第二卷的校样时,对由计算机排版的校样的低质量感到无法忍受。因此
决定自己来开发一个高质量的计算机排版系统,这样就有了 TeX。TeX 的输出文件称为 DVI 文件,即是“Device
Independent”。一旦 TeX 处理了你的文件,你所得到的 DVI 文件就可以被送到任何输出设备,如打印机、屏幕
等并且总会得到相同的结果,而这与这些输出设备的限制没有任何关系。这说明 DVI 文件中所有的元素,
从页面设置到文本中字符的位置都被固定,不能更改。


\faq{什么是 TDS?}{what-its-tds}

TDS 全称 \TeX{} Directory Structure,意为 \TeX{} 目录结构,即 \TeX{} 发行版的文件组织结构。大部分
\TeX{} 发行版都将自身的文件组织成相近的路径结构,也就是 TDS。TDS 也称为 TEXMF 树,这是 \TeX{} 与
\METAFONT{} 的合称。很多系统的 TDS 结构都以 |texmf| 或者类似的词作为 TEXMF 树的根目录名,如在
\TeXLive{} 中,安装目录下的 |texmf-dist|、|texmf-var| 等就是两个不同的 TEXMF 树,
如图~\ref{fig:texmf-dir}。

\begingroup
  \dirtree {%
    .1 TEXMF树.
    .2 bibtex/\DTcomment{\BibTeX{} 相关文件}.
    .3 bib/\DTcomment{公用 bib 数据库}. % TODO: verb 有问题
    .3 bst/\DTcomment{格式文件}.
    .2 doc/\DTcomment{各类用户文档}.
    .3 bibtex/\DTcomment{\BibTeX{} 相关文档}.
    .3 fonts/\DTcomment{字体文档}.
    .3 generic/\DTcomment{通用于各种格式的文档}.
    .4 pgf/.
    .5 pgfmanual.pdf\DTcomment{PGF/\TikZ{} 用户手册}.
    .3 latex/\DTcomment{用于 \LaTeX{} 格式的文档}.
    .4 ctex/.
    .5 ctex.pdf\DTcomment{\CTeX{} 宏集用户手册}.
    .5 README.md\DTcomment{\CTeX{} 宏集简短介绍}.
    .3 texlive/\DTcomment{\TeXLive{} 发行版自身的文档}.
    .2 font/\DTcomment{字体相关文件}.
    .3 opentype/\DTcomment{OpenType 格式的字体}.
    .3 source/\DTcomment{字体源代码}.
    .3 truetype/\DTcomment{TrueType 格式的字体}.
    .3 type1/\DTcomment{Type1 格式的字体}.
    .2 scripts/\DTcomment{可执行脚本}.
    .3 l3build/\DTcomment{\LaTeX{} 构建、测试脚本}.
    .3 latexmk/\DTcomment{自动编译系统}.
    .3 texdoc/\DTcomment{文档查询系统}.
    .2 source/\DTcomment{源代码}.
    .3 bibtex/\DTcomment{\BibTeX{} 相关宏包代码}.
    .3 fonts/\DTcomment{字体源代码}.
    .3 generic/\DTcomment{通用于各种格式的宏包代码}.
    .3 latex/\DTcomment{用于 \LaTeX{} 格式的宏包代码}.
    .4 ctex/\DTcomment{\CTeX{} 宏集源代码}.
    .5 ctex.dtx.
    .5 ctex.ins.
    .5 ctexpuct.spa.
    .2 tex/\DTcomment{\TeX{} 宏,可被引擎读入}.
    .3 generic/\DTcomment{通用于各种格式}.
    .3 latex/\DTcomment{用于 \LaTeX{} 格式}.
    .4 base/\DTcomment{\LaTeX{} 的基本宏文件}.
    .5 article.cls.
    .5 book.cls.
    .5 report.cls.
    .5 latex.ltx.
    .4 beamer/\DTcomment{\cls{beamer} 宏集相关文件}.
    .4 ctex/\DTcomment{\CTeX{} 宏集相关文件}.
    .5 ctexart.cls.
    .5 ctexbeamer.cls.
    .5 ctexbook.cls.
    .5 ctexrep.cls.
    .5 ctex.sty.
    .3 plain/\DTcomment{用于 \PlainTeX{} 格式}.
    .3 xetex/\DTcomment{用于 \XeTeX{} 引擎}.
    .3 xelatex/\DTcomment{用于 \XeTeX{} 引擎下的 \LaTeX{} 格式}.
    .4 xecjk/\DTcomment{\pkg{xeCJK} 宏包相关文件}.
    .5 xeCJK.sty.
  }
  \captionof{figure}{TEXMF 树目录结构}
  \label{fig:texmf-dir}
\endgroup
(来自刘海洋《LaTeX 入门》)

%\faq{从TeX编写(文本)文件}{tex-text-write}

%\faq{\\special命令}{special-command}

\faq{Overleaf 是什么?如何使用?}{overleaf}

\href{https://www.overleaf.com}{Overleaf} 是一个在线的 \LaTeX{} 协作平台,可在各大浏览器内运行,
也就是说,不需要本地安装 \LaTeX{} 套装,就可以撰写编译 \LaTeX{} 文件。2017 年 Overleaf 与另一家
在线 \LaTeX{} 平台 \href{https://www.sharelatex.com}{ShareLaTeX} 合并,新的
\href{https://v2.overleaf.com}{Overleaf v2 平台} 囊括了两个原先平台的主要功能,而原有的 Overleaf 和
ShareLaTeX 用户可以在 Overleaf v2 上继续编辑原先所有的项目文件。

由于个别地区、单位的网络问题或限制,有些用户可能无法顺利登录、访问 Overleaf,在此我们不作
(也不方便)讨论。

Overleaf v1 目前运行的是 \TeXLive{} 2016,Overleaf v2 则是 \TeXLive{} 2017。升级 \TeXLive{} 2018
可能需要等到 2019 年。Overleaf v1 和 v2 都支持 \LaTeX{} + dvipdf、\pdfLaTeX{}、\XeLaTeX{}、
\LuaLaTeX{} 等多种编译方式;通过 |latexmkrc| 设置也可用 \pTeX{} 等其它编译器。支持中文,可以用
传统手段 \pkg{CJK} 宏包
\footnote{示例见 \url{https://www.overleaf.com/read/xhzmbjjnbfrq}。},
也可用 \CTeX{} 宏集的新一代解决方案
\footnote{示例见 \url{https://www.overleaf.com/read/gndvpvsmjcqx}。},
当然我们比较推荐后者。由于字库版权问题,Overleaf 上不会有 Windows 中文字体可用,因此如果一定要用
Windows 中文字体的话(高校论文要求等原因),则还是建议本地编译。毕竟字库文件太大,逐个上传还是
比较麻烦。

Overleaf 用的服务器系统是 Ubuntu Linux,因此文件名区分大小写。编译时用的是 \pkg{latexmk},因此在
Overleaf上每“编译”一次,其实可能已经运行了 4 次 |xelatex| + 1 次 |bibtex| + 1 次 |makeglossaries|
等等,把所有交叉引用、文献引用等等都解决掉,也可以自动调用 |biber| 或 \pkg{biblatex},或自动编译
\pkg{chapterbib}、\pkg{multibib} 所导出的多个文献列表。

Overleaf 编译也自动启用 |--shell-escape|,因此 \pkg{minted}、\pkg{tikz-externalize} 
等都可以使用。|--shell-escape| 本来是有安全隐忧的,但是每个 Overleaf 文件项目都是一个单独的 Docker 
容器(container),因此若是有谁想玩什么把戏,也不会波及其它项目。

【其它重点功能介绍明天,不,有空时再补充:multi-authors collaboration, RichText mode,
Overleaf Gallery and templates, one-click submission to journals, git bridge and github sync,
history tracking, 文件项目空间上限】

如果对 Overleaf 有任何问题,建议直接发送电子邮件至
\href{mailto:support@overleaf.com}{support@overleaf.com} 询问,不必过多自行揣测。若是担心语言沟通
问题,大可放心——Overleaf 技术支援人员有熟悉中文的。

%\pkg{minted} 在 Overleaf v1、v2 都是可以用的,例子参照
%\url{https://www.overleaf.com/read/qphhfvnsddbs}。MATLAB 高亮也一样可以用 \pkg{minted} 完成,注意
%字母大小写:
%
%\begin{verbatim}
%\begin{minted}{matlab}
%...
%\end{minted}
%\end{verbatim}
%
%Overleaf v1 的确有文件数目及项目大小限制,视收费与否而定:\url{https://www.overleaf.com/help/297};
%Overleaf v2 则相对宽松得多:所有项目(包括免费)都无文件大小上限,条件:每个项目限 2000 文件,以及
%可编辑纯文本文件不大于 5MB。详见
%\url{https://www.sharelatex.com/learn/Uploading_a_project#Limitations_on_Uploads}。
%
%Overleaf v1 是支持两层及以上的文件夹的,详情看 \url{https://www.overleaf.com/help/187};
%Overleaf v2 可以直接右键资料夹,再创建子资料夹。
%
%Overleaf v2 目前还是比较稳定的(2018 年 9 月将会全面上线)。当然如果是特定地区、单位限制的原因,
%则不方便在这里讨论。
%
%【这里先预定个位置,以后可以写个篇幅比较长的功能介绍。若对 Overleaf 有任何问题,建议直接电邮
%support@overleaf.com 询问,不必过多自行揣测。若是担心语言问题,Overleaf 技术支援人员有谙中文的。】

\faq{编译器与编辑器的区别是什么?}{ide-compile-diff}

在 \pkg{lshort} 中,确切解释了,所谓编译器,真正的名称叫排版引擎,是读入源代码并编译生成文档的
程序,如 \pdfTeX{}、\XeTeX{} 等。

编辑器,其实是用户书写源代码的工具,例如 Windows 下的记事本、Ubuntu 下的 gedit 等等。目前很多编辑器
都提供了“编译”按钮,本质上是基于命令行调用了编译器。

\faq{未来有计划做到像 Bakoma 那样所见即所得吗?}{bakoma}

\LaTeX{} 的缺点之一,相比“所见即所得”的模式有一些不便,为了查看生成的文档,用户总要不停地编译。  

% \faq{什么是“决议”(resolutions)}
% \faq{什么是(\TeX{})宏}
% \faq{什么是\LaTeX{}类和工具包}
% \faq{什么是PK文件}
% \faq{什么是TFM文件}
% \faq{什么是编码}
% \faq{什么是EC字体}
% \faq{什么是虚拟字体}
% \faq{什么是“Encapsulated PostScript”(EPS)}
% \faq{什么是DVI驱动程序}
% \faq{什么是“Berry命名方案”}
