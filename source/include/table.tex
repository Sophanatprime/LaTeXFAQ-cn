% Copyright (C) 2018 by latexstudio <http://www.latexstudio.net>
%
% This program is free software: you can redistribute it and/or modify
% it under the terms of the GNU General Public License as published by
% the Free Software Foundation, either version 3 of the License, or
% (at your option) any later version.
%
% This program is distributed in the hope that it will be useful,
% but WITHOUT ANY WARRANTY; without even the implied warranty of
% MERCHANTABILITY or FITNESS FOR A PARTICULAR PURPOSE.  See the
% GNU General Public License for more details.
%
% You should have received a copy of the GNU General Public License
% along with this program.  If not, see <http://www.gnu.org/licenses/>.
%

\section{表格篇}


\faq{如何指定表格的总宽度}{}

可以看看tabularx、tabu等宏包。


\faq{指定列宽度的表格如何使单元格内容居中}{}

指定宽度的表格列一般采用 p\{\}
形式的列格式,这种列格式下,表格内容是两端对齐的,如果想使其成为居中对齐需要借助
array 宏包提供的功能,示例如下:

\begin{verbatim}
\begin{tabular}{c|>{\centering\arraybackslash}p{4cm}}
  \hline
  1  &  3.530  \\
  2  &  456.0  \\
  3  &  78.945 \\
  4  &  3.65   \\
  \hline
\end{tabular}
\end{verbatim}

而 |>{}p{}|
这样的格式在文档的应用过程中是非常不方便的,array 宏包同时提供了
\cs{newcolumntype} 宏命令可以将其定义为一个较为简短的格式,如:

\begin{verbatim}
\newcolumntype{z}[1]{>{\centering\arraybackslash}p{#1}}
\end{verbatim}

从而可以在正文中使用

\begin{verbatim}
\begin{tabular}{c|z{4cm}}
\hline

1  &  3.530  \\
2  &  456.0  \\
3  &  78.945 \\
4  &  3.65   \\
\hline
\end{tabular}
\end{verbatim}

类似的,采用 \cs{raggedright} 或
\cs{raggedleft} 替换\cs{centering} 可以使得单元格内容变成左对齐或右对齐。


\faq{tabularx 中的 X 列格式如何居中对齐}{}

同样采用 array 宏包的 |>|\marg{format} 方法,并利用
\cs{newcolumntype} 定义新的列格式,如:

\begin{verbatim}
\usepackage{array,tabularx}  % this line in preamble
\newcolumntype{Z}{>{\centering\arraybackslash}X} % this line in preamble
\begin{tabularx}{\linewidth}{ZZ}
\hline

1  &  3.530  \\
2  &  456.0  \\
3  &  78.945 \\
4  &  3.65   \\
\hline
\end{tabularx}
\end{verbatim}


\faq{tabularx 中的 X 列格式,当单元格内容发生换行时,如何使同一行其他列的单元格垂直居中对齐?}{}

对于指定宽度的表格列格式
p\{\},单元格内一旦进行换行,该单元格同一行内其他列的单元格内容均为垂直方向上顶端对齐,我们可以使用
array 宏包,以 m\{\} 列格式或者 b\{\} 列格式 替代 p\{\}
格式即可实现垂直居中对齐或垂直底部对齐。对于 tabularx 中的 X
列格式,也是采用同样的思路实现,只是这里需要对
\cs{tabularxcolumn} 宏进行重定义如下:

\begin{verbatim}
\usepackage{array,tabularx}   % this line in preamble
\renewcommand{\tabularxcolumn}[1]{m{#1}}  % this line in preamble
\end{verbatim}

以上则将同行的其他列单元格设置为垂直居中对齐。显然的,垂直底部对齐的设置方法是将重定义宏命令中的
m\{\#1\} 替换为 b\{\#1\} 即可。


\faq{booktabs的三线表,竖线为什么是不连续的?}{}

宏包的作者为表格线的前后都增加了额外的sep,而且,宏包的作者认为三线表是不应该有竖线的。当然,如果你一定想要使用竖线,不妨以下面两个命令将表格线前后的sep设置为0pt。

\begin{verbatim}
\usepackage{booktabs} % this line in preamble
\setlength{\belowrulesep}{0pt}
\setlength{\aboverulesep}{0pt}
\end{verbatim}


\faq{表格的一列全是公式,有什么办法能输入简单些?}{}

可以使用 array 宏包,|>{}| 与|<{}|
可以为一列数据前后加上特定的宏命令。在一列数据前后均加上 |$| 则把这列数据放入数学模式中,举例如下:
\begin{verbatim}
\usepackage{array} % this line in preamble
\begin{tabular}{>{$}c<{$} c}
  \hline
  \multicolumn{1}{c}{function} & value \\
  g(x)                         & 3.65  \\
  f(x)                         & 2.58  \\
  \sin(x)                      & 14.7  \\
  \hline
\end{tabular}
\end{verbatim}

第一列数据省去了输入数学模式起止符号 |$| 的痛苦。对于不需要放入数学模式的单元格,比如表头,需要用 |\multicolumn{1}{c}{xxx}| 的方式来保护一下,重新指定对齐方式。


\faq{我的表格单元格内容是一个列表环境 (enumerate/itemize),它和表格横线之间间距好大啊,怎么能把这些间距去掉?}{}

把列表环境放入到 minipage 环境中即可,即使表格列格式采用的是p{<width>}格式。


\faq{如果想让表格中数字小数点对齐要怎么做}{}

可以借助 @ 的功能,如

\begin{verbatim}
\begin{tabular}{r@{.}l}
  \hline
  1 & 0 \\
  23 & 1 \\
  \hline
\end{tabular}
\end{verbatim}

或者借助 warpcol 宏包提供的功能,如
\begin{verbatim}
\documentclass{article}
\usepackage{warpcol}
\begin{document}
\begin{tabular}{P{3.1}P{-2.1}}
  \hline
  \multicolumn{1}{c}{Label 1} & \multicolumn{1}{c}{Label 2} \\
  \hline
  123.4 & -12.3 \\
  12.3 & 12.3 \\
  1.2 & 1.2 \\
  \hline
\end{tabular}
\end{document}
\end{verbatim}

还可以借助 array 和 dcolumn 的配合,如

\begin{verbatim}
\documentclass{article}
\usepackage{array,dcolumn}
\newcolumntype{d}[1]{D{.}{.}{#1}}
\begin{document}
\begin{tabular}{cd{3}}
  \hline
  1 & 3.14 \\
  2 & 27.12 \\
  3 & 78.095 \\
  \hline
\end{tabular}
\end{document}
\end{verbatim}


还可以借助 array 和 dcolumn 的配合,如

\begin{verbatim}
\documentclass{article}
\usepackage{array,dcolumn}
\newcolumntype{d}[1]{D{.}{.}{#1}}
\begin{document}
\begin{tabular}{cd{3}}
  \hline
  1 & 3.14 \\
  2 & 27.12 \\
  3 & 78.095 \\
  \hline
\end{tabular}
\end{document}
\end{verbatim}


\faq{表格竖排}{}

\begin{verbatim}
\documentclass{ctexart}
\usepackage[usestackEOL]{stackengine}

\begin{document}

\setlength\normalbaselineskip{11pt}
\strutlongstacks{T}
\begin{tabular}{|c|c|c|}
\hline
Foo bar & {\Centerstack{ 这 \\ 一 \\ 列 \\ 竖 \\ 排 }} & Foo bar \\
\hline
\end{tabular}

\end{document}
\end{verbatim}


\faq{跨页长表格}{}

\begin{verbatim}
\usepackage{longtable}
\end{verbatim}

,做好对长表格跨页时的设置


\faq{双栏中表格过大怎么调整?}{}

\begin{itemize}

\item
  方法一:用 graphicx 宏包提供的 \cs{resizebox} 命令:
\end{itemize}

\begin{verbatim}
\resizebox{width}{height}{function}
\end{verbatim}

resizebox 会放缩 function 中的内容到 width 宽度、height
高度。需要注意的是,同时指定宽度和高度,一般会导致缩放的内容变形,你也可以指定其中一项,\sout{另一个用!占位,}这样系统会自适应另一个参数,即相当于scale命令。
* 方法二:用 \texttt{table*} 取代 table 环境,针对的是单栏表格。 *
方法三:将表格中的字体缩小。 * 方法四:使用横排:使用 rotating 宏包


\faq{如何制作列数可变的表格,例如试卷的计分表?}{}

主要是使用 makecell 和 interfaces-makecell 宏包。下面给出一个 MWE。

\begin{verbatim}
\documentclass{standalone}
\usepackage{ctex,calc,makecell,interfaces-makecell,CJKnumb,tabularx,multirow}

\newcounter{TotalPart}
\newcounter{SubColumn}
\newcounter{EmptyColumn}

\setcounter{TotalPart}{1}

% 计分表制作
\newcommand{\ScoreTable}{
  \setcounter{SubColumn}{\value{TotalPart}+2}
  \setcounter{EmptyColumn}{\value{TotalPart}+4}
  \begin{tabularx}{\textwidth}{|*{\theSubColumn}{X<{\centering}|}*{3}{c|}}
    \hline
      \multicolumn{\theSubColumn}{|c|}{\multirow{2}{*}{试卷卷面成绩}}
      & \multicolumn{1}{c|}{\multirow{3}{3em}{课程考核成绩占~\%}}
      & \multicolumn{1}{c|}{\multirow{3}{3em}{平时成绩占\,\%}}
      & \multicolumn{1}{c|}{\multirow{3}{3em}{课程考核成绩}}
    \\
      \multicolumn{\theSubColumn}{|c|}{}
      & & &
    \\
    \cline{1-\theSubColumn}
      \hfill 题 \hfill 号 \hfill~
      & \repeatcell{\theTotalPart}{text=\CJKnumber{\column}}
      & \hfill 小 \hfill 计 \hfill~
      & & &
    \\
    \hline
      \hfill 得 \hfill 分 \hfill~
      & \eline{\theEmptyColumn}
    \\
    \hline
  \end{tabularx}
}

\begin{document}

\ScoreTable

\end{document}
\end{verbatim}

CJKnumb 宏包是为了把阿拉伯数字转换为小写汉字序号。 calc
宏包是为了做四则运算。 tabularx 宏包是为了做列宽自动扩展的表格。
multirow 宏包是为了合并单元格。 makecell 是制作表格。
interfaces-makecell
宏包提供了一系列命令,使得制作可变表格称为可能,同时简化了表格制作。


\faq{如何固定表格的总宽度?}{}

使用 tabular* 环境或 tabularx
宏包提供的同名环境即可固定表格的总宽度,宏包 tabu
功能更为强大,用法也更为复杂,可参见相应宏包文档说明。


\faq{表格在单元格内如何换行?}{}

可以通过限制列宽实现,例如下面的例子

\begin{verbatim}
\begin{tabular}{|c|c|m{50mm}|}%这里用m则必须调用array宏包
  \hline
  a & b & \LaTeX{}表格固定列宽自动换行自动换行自动换行自动换行自动换行\\
  \hline
  a & b & \LaTeX{}表格固定列宽自动换行自动换行自动换行自动换行自动换行\\
  \hline
  a & b & \LaTeX{}表格固定列宽自动换行自动换行自动换行自动换行自动换行\\
  \hline
\end{tabular}
\end{verbatim}


\faq{如何插入子图/表,各自分别带子标题,不带子标题?}{}

可参见并列图形、并列子图的排列


\faq{如何减小表格,插图,公式,列表等前后空白?}{}

表格、插图、公式、列表的前后空白很多是由于不良的文本结构引起的,比如太短篇幅的正文,接连几级标题之间没有正文内容,甚至标题之间只有插图和表格等浮动体而没有任何说明的正文,这些都是不好的行文习惯,应杜绝这样的行文方式。此外,一些不良的代码写法也会引入较大的空白,如:

\begin{verbatim}
\begin{center}
  \begin{figure}
  ...
  \end{figure}
\end{center}
\end{verbatim}

或者

\begin{verbatim}
\begin{figure}
  \begin{center}
  \includegraphics{x.pdf}
  \caption{the title}
  \end{center}
\end{figure}
\end{verbatim}

而应该采用的方式是:

\begin{verbatim}
\begin{figure}
\centering
\includegraphics{x.pdf}
\caption{the title}
\end{figure}
\end{verbatim}

这是因为 center 环境本身就是一个 list
列表环境,其与上下文之间就有垂直间距,加上figure
浮动体与正文之间的间距,插图与正文之间的间距自然就变大了。


\faq{表格如何分页?}{}

这个问题可见跨页长表格。


\faq{表格怎样可以旋转90度?}{}

希望旋转90度的表格多半是由于过宽而需要进行横排,这里一个方法是使用
rotating 宏包,使用方法非常简单,用 sidewaytable 替代 table
即可,但这种表格不能实现跨页长表格(当然又宽又长的表格确实很少见);另一个方法是使用lscape
宏包提供的 landscape
环境,进入横排状态,在其中使用相应的环境即可,这种方法可以实现跨页表格,但进入和退出landscape环境时总是会新开一页再进行排版,因此,在其之前的页面可能会留有大量的空白。两种方法各有利弊,可以根据实际需要进行选择。


\faq{如何使用图表目录?}{}

\cs{listoftables}
\cs{listoffigures}


\faq{图表如何使用双语标题}{}

使用 bicaption 宏包或 ccaption 宏包。


\faq{如何产生表格的竖线,在模板的三线表中产生竖线?}{}

竖线的产生与否与表格的环境无关,在定义表格列时以 \textbar{}
分隔列格式即可产生竖线。
\begin{verbatim}
\begin{tabular}{l|c|r|}
  ...
\end{tabular}
\end{verbatim}


\faq{如何在长表格\{longtable\}环境中设置文字自动换行或者固定列宽?}{}


\faq{如何实现表格的奇偶行不同的颜色,长表格也要适用。}{}


\faq{如何使表格单元的左对齐?}{}

不知道这个问题是啥意思。。。
难道不是在列格式中选择lcr分别表示左中右么?

\faq{表格中如何划对角线?}{}

有宏包 slashbox 或 diagbox 可以制作表格对角线,不过slashbox
由于没有明确的自由许可信息,已经不为 TeXLive
所收录了。一个好消息是:diagbox
有中文版的说明文档,作者的说明总比这里的说明更为准确,直接查阅宏包文档是更好的选择。
