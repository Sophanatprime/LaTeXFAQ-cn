% !TEX root = ../latex-faq-cn.tex

\section{安装与配置问题}
\label{sec:install}

\faq{如何下载 \TeXLive{}?}{download-texlive}

\TeXLive{} 分为在线安装和 ISO 镜像安装两种方式。在线安装时,需保证网络连接良好,一旦网络异常,很有
可能前功尽弃。因此对于普通用户,我们建议先行下载 \TeXLive{} 的 ISO 镜像文件至本地,再进行安装。

\TeXLive{} 的 ISO 镜像文件可以在这一网址找到:\url{http://mirror.ctan.org/systems/texlive/Images/}。
打开链接后,您可能会找到若干个 ISO 文件,如 \file{texlive.iso}、\file{texlive2018.iso}、
\file{texlive2018-20180414.iso} 等,除了名称的差别,它们是完全一样的。

如果需要进行在线安装,可以打开这一网址:\url{http://mirror.ctan.org/systems/texlive/tlnet/}。
其中包含的这些文件可用于安装 \TeXLive{}:

\begin{itemize}
  \item \file{install-tl-windows.exe}:适用于 Windows 系统下的安装程序,可直接双击打开;
  \item \file{install-tl-unx.tar.gz}:适用于类 Unix 系统,解压后需要以脚本方式执行安装;
  \item \file{install-tl.zip}:同时支持 Windows 和类 Unix 系统,解压后同样需要以脚本方式安装。
\end{itemize}

Linux各发行版的包管理系统中一般都包括\TeXLive{},一些发行版更新速度也较快,可以用
其自身的包管理命令安装而无需下载镜像文件。如Debian类系统(包括Ubuntu)可以通过以下命令安装
texlive完全版本及与其相关的依赖包:
\begin{shcode}
  aptitude install ~n^texlive
\end{shcode}


\faq{如何安装 \TeXLive{}}{install-texlive}

\TeXLive{} 为用户提供了官方安装手册,其中文版地址是:
\url{https://tug.org/texlive/doc/texlive-zh-cn/texlive-zh-cn.pdf}
建议有余力的用户通读手册,以了解更多内容。

也可以参照以下网址 \url{https://www.tug.org/texlive/quickinstall.html}

\faq{如何下载 \MiKTeX{} 安装包?}{install-miktex}

访问以下链接即可 \url{https://miktex.org/}

\faq{如何下载 \MacTeX{} 安装包?}{install-mactex}

\MacTeX{} 安装包下载地址 \url{http://mirror.ctan.org/systems/mac/mactex/MacTeX.pkg}


\faq{如何下载 pro\TeX{}t 安装包?}{download-protext}

访问以下链接即可 \url{http://mirror.ctan.org/systems/protext/protext.exe}


\faq{如何挂载镜像文件:}{mount-iso}

目前市面上有很多虚拟光驱软件可供用户选择,例如 UltraISO。

特别一提,在 Windows 8、Windows 10
操作系统中,默认被双击后,镜像文件将会直接挂载。

在 Linux 操作系统中,可使用命令行挂载镜像文件:
\begin{shcode}
  mount -o loop ~/download/texlive.iso ~/iso
\end{shcode}


\faq{挂载镜像文件后该如何做?}{after-mount}

windows 用户可以双击 \file{install-tl-windows.bat} 文件来进行安装。

Linux 用户请在命令行执行 |./install-tl| 进入 no gui 安装模式


\faq{双击 install-tl-windows.bat 出现错误怎么办?}{error-install}

使用命令行。同时按下 win 键和 R 键,打开“运行”窗口,在窗口的“打开”处,
输入 cmd 打开命令行窗口(黑窗)。

在黑窗内输入
\begin{shcode}
  cd /d ~
\end{shcode}
后按 Enter 键(即执行该命令),此处 |~| 代指 \file{install-tl-windows.bat} 
所在目录,
例如 \file{C:/Downloads} 等,注意命令中的空格。

进入目录后继续执行
\begin{shcode}
  install-tl-windows.bat -no-gui
\end{shcode}
开启纯命令行安装模式。默认状态下,点击 I 键,安装便会开始。
若用户想改变安装路径或其他设置,只需根据屏幕提示进行更改即可。
特别强调,安装路径一定是不带空格的纯英文路径。


\faq{使用命令行安装 \TeXLive{} 出现 goodbye 怎么办?}{error-goodbye}

主要是因为缺少环境变量。将 \verb|C:\Windows\system32| 添加到系统环境变
量中即可。或者在命令行中执行
\begin{shcode}
  path=%path%;C:\Windows\system32
\end{shcode}
后再尝试安装。也可能由于下载文件损坏而造成安装失败,这时应重新下载。


\faq{想在 Linux 系统中使用 GUI 模式安装该怎么做?}{linux-gui}

自行安装 perl,详细办法请上网自行搜索。然后执行命令
\begin{shcode}
  ./install-tl -gui wizard
\end{shcode}
或
\begin{shcode}
  ./install-tl -gui perltk
\end{shcode}


\faq{如何配置 \TeXLive{} 的环境变量?}{texlive-path}

Windows 用户一般不必担忧这个问题。因为 \TeXLive{} 已经自动将环境变量写入,用户不必自己手动修改。

Linux 用户需要手动配置环境变量。例如,将
\begin{shcode}
  TEXDIR=/usr/local/texlive/2018
  if [ -d $TEXDIR ]; then
    export PATH="$TEXDIR/bin/x86_64-linux:$PATH";
    export MANPATH="$TEXDIR/texmf-dist/doc/man:$MANPATH";
    export INFOPATH="$TEXDIR/texmf-dist/doc/info:$INFOPATH";
  fi;
\end{shcode}

写入 \file{~/.profile}。注意本例中的 2018 可以根据需要修改。
例如部分用户还在使用 \TeXLive{} 2017,就可将 2018 改为 2017 等等。


\faq{如何判断 \TeXLive{} 安装成功?}{texlive-success}

在新打开的命令行窗口中执行
\begin{shcode}
  tex -v
\end{shcode}
若命令行窗口中显示“TeX Live 2018”等内容,即说明安装成功。


\faq{如何删除 \TeXLive{}}{uninstall-texlive}

Windows 用户请找卸载批命令文件,如 |C:\texlive\2018\tlpkg\installer\uninst.bat|

Linux 用户请直接删除文件夹,如执行
\begin{shcode}
  rm -rf /usr/local/texlive/2018
  rm -rf ~/.texlive2018
\end{shcode}
并且手动清理环境变量。

\faq{\TeXLive{} 如何升级宏包?}{texlive-update}

建议使用命令行升级宏包。

首先指定源,执行命令
\begin{shcode}
  tlmgr option repository ~
\end{shcode}
这里的 \verb|~| 指代地址,如 \verb|ctan| 即为让系统自动寻源;
\verb|http://mirrors.tuna.tsinghua.edu.cn| 
\verb|/CTAN/systems/texlive/tlnet| 即为手动指定清华的源。

接下来,执行命令
\begin{shcode}
  tlmgr update --self
\end{shcode}
升级 tlmgr 本身。 然后,我们就可以升级宏包了。实际上,tlmgr
升级所有宏包的代码非常简单,执行命令
\begin{shcode}
  tlmgr update --all
\end{shcode}
升级过程中如果遇到问题,可以执行
\begin{shcode}
  tlmgr update −−reinstall−forcibly−removed --all
\end{shcode}

\faq{在 Ubuntu 系统中使用升级宏包,系统显示未找到该命令该如何做}{ubuntu-tlmgr}
首先,执行命令
\begin{shcode}
  sudo visudo
\end{shcode}
打开一个窗口,窗口中找到
\begin{shcode}
  Defaults secure_path="/usr/local/sbin..."
\end{shcode}
将其改为(假设是64位操作系统)
\begin{shcode}
  Defaults secure_path="/usr/local/texlive/2018/bin/x86_64-linux:/usr/local/..."
\end{shcode}
然后按 \verb|ctrl+X| 保存退出即可。

\faq{\MiKTeX{} 如何升级宏包}{miktex-update}

\MiKTeX{}
可以用界面升级宏包,也可以使用命令行升级宏包:
\begin{shcode}
  mpm --admin --update
\end{shcode}

有些用户经常升级失败是因为源不稳定造成的。建议到
\url{https://miktex.org/pkg/repositories} 找稳定的源。
指定更新源地址的命令是
\begin{shcode}
  mpm --admin --set-repository=http://.../CTAN/systems/win32/miktex/tm/packages
\end{shcode}

\faq{如何自动升级 \TeXLive{} 宏包?}{texlive-autoupdate}

\href{http://pd10ibe5c.bkt.clouddn.com/TeXLive\%E5\%AE\%8F\%E5\%8C\%85\%E6\%AF\%8F\%E6\%9C\%88\%E8\%87\%AA\%E5\%8A\%A8\%E6\%9B\%B4\%E6\%96\%B0.zip}{这里可以下载每月自动升级\TeXLive{}宏包的脚本}。
\href{http://htharoldht.com/texlive-package-automatically-upgrades-every-month/}{这里是该脚本的说明}。


\faq{不同平台 \LaTeX{} 编辑器推荐}{editors}

用户编写的 tex 文件,本质上是文本文件,因此很多编辑器都可以对 tex 文件进行更改。
某些编辑器,如 Notepad++,VSCode,Sublime Text 等,还对 tex 文件进行了语法高亮,
甚至可以利用插件做成一个 IDE。

TeXworks 是集成在 \TeXLive{} 和 \MiKTeX{} 中的编辑器(\MacTeX{} 则集成了类似的 
TeXShop),
轻量简洁,适合新手学习。

TeXStudio 是一款跨平台的开源 \LaTeX{} IDE(集成开发环境)。
对于大部分用户而言,它的功能足以满足需要,下载可访问 
\url{https://github.com/texstudio-org/texstudio/releases}。

Texmaker 是一款免费、现代、跨平台的 \LaTeX{} 编辑器。 它能够在 Linux,macOS 和 
Windows 系统中使用,
并且将很多开发 \LaTeX{} 文件的工具集成在了一个应用当中。
详情见官网:\url{http://www.xm1math.net/texmaker/}。

WinEdt 是一款功能强大且多样的 Windows 专用文本编辑器,具有很强的创建和编辑 \LaTeX{} 
文档的能力,
可与TeX系统(如\MiKTeX{}或 
\TeXLive{})无缝集成。详情见官网:\url{http://www.winedt.com/}。

Texpad 是运行于 macOS/iOS 在线平台的编辑器,带自动编译,支持多人联合编辑,更多内容可
访问
\url{https://www.texpad.com}

Visual Studio Code(vscode),是一款强大的跨平台编辑器。
安装 LaTeX Workshop 插件即可进行编译工作。Visual Studio Code 官网见 
\url{https://code.visualstudio.com/}。
配置可参考 \href{https://github.com/EthanDeng/vscode-latex}{LaTeX 编译环境配置:Visual Studio Code 配置简介} 和 \href{https://github.com/James-Yu/LaTeX-Workshop/wiki}{LaTeX-Workshop 的 Wiki}。


\faq{如何在 Sublime 上配置 \LaTeX{} 编译环境}{sublime-latex}
可以参考 \href{https://github.com/EthanDeng/sublime-text-latex}{Sublime Text 搭建 LaTeX 编写环境}。


\faq{\LaTeX{} 文档能转成 Microsoft Word 格式吗}{tex2word}

严格来讲,可以做,例如利用 pandoc 等工具进行自动转换,但结果往往不尽如人意,因此十分不
建议这样做。


\faq{新手应该选择什么发行版,什么编辑器最省心}{tex-beginner}

其实选择什么发行版都可以啦,只不过大家说的最多的是 \TeXLive{},其次是 \MiKTeX{}。
编辑器也随意,像 \TeXLive{} 和 \MiKTeX{} 里自带的 TeXworks,
第三方的 TeXMaker,TeXStudio 等都是免费的编辑器。
有付费习惯的 Windows 用户也可以选择 WinEdt。
Mac 用户通常使用的是 \MacTeX{},它里面集成了 TeXShop 编辑器。


\faq{清华大学的 \TeXLive{} 镜像没有其他语言版?}{}

不仅清华大学的没有,其他镜像的也没有。\TeXLive{} 本身不存在语言的问题。
对于一般用户而言,能通过命令行调用 \TeXLive{} 的引擎的人都不多,命令行需要尽量避开中文。
你可能好奇的是编辑器如何变成中文。这个需要看编辑器本身的设置。


\faq{TeXStudio 怎么自定义快捷键?}{}

options $\to$ configure texstudio $\to$ shortcuts


\faq{有哪些支持实时刷新的 pdf 阅读器?}{}

\begin{itemize}
  \item Windows 下可以使用 \href{https://www.sumatrapdfreader.org/free-pdf-reader.html}{SumatraPDF}
  \item Mac 下可以使用 \href{http://skim-app.sourceforge.net/}{Skim}。
  \item Linux 下的 \href{https://gitlab.gnome.org/GNOME/evince}{Evince} 和
  \href{https://okular.kde.org/}{Okular} 也支持实时更新 pdf。
\end{itemize}


\faq{不同编辑器和不同 pdf 阅读器如何设置正反向搜索?}{}

配置 Sublime Text:
\begin{itemize}
  \item
    手写编译命令:
    % TODO: 如何更好地换行显示
\begin{jsoncode}
  {
      "shell_cmd": "xelatex -synctex=1 \"${file}\" && evince \"$file_base_name.pdf\"",
  }
\end{jsoncode}
    说明: 添加 |-synctex=1| 参数生成 \file{synctex.gz} 文件,以支持正反向搜索。
    
    |evince "$file_base_name.pdf"| 用 Evince 文档查看器打开生成的 PDF 文档,
    或者你可以换成其它 PDF 查看器。
    
    |$file_base_name| 获取不包括后缀的文件名。
    两条命令需要前面执行完正确再执行后面,用 |&&| 分隔开。
  \item
    通过 Package Control 安装插件 LaTeXTools。
    
    TeXStudio不用配置,只需要按住 Ctrl,鼠标左键分别点击代码窗口和内置 pdf 阅读器页面,
    会分别定位到 pdf 和代码窗口。
\end{itemize}


\faq{使用 \pkg{minted} 之前要如何配置环境?}{}

详细的配置在 \pkg{minted} 宏包文档中有介绍。
安装 python,选定安装 pip。打开命令行,执行
\begin{shcode}
  setx path=%path%;[Python];[\Python\Scripts]
\end{shcode}
这里的 |[Python]| 和 |[\Python\Scripts]| 代指你安装 python 的路径和该路径下的 scripts 文件夹。
如 |D:\Python\Python36| 和 |D:\Python\Python36\Scripts|。
然后在命令行中执行
\begin{shcode}
  pip install Pygments
\end{shcode}
进行安装。
最后在编译文档的时候添加 |-shell-escape| 选项。


\faq{\TeXLive{}为什么要采取每年一个大版本的制度?}{}

参考  \url{https://tex.stackexchange.com/questions/107017/why-does-tex-live-require-yearly-updates}


%\faq{在经常使用某个非 CTAN 收录的宏包的情况下。怎样安装这种非字体宏包?}{}


%\faq{怎样安装字体宏包?}{}


\faq{为了预防 \TeX{} 源文件用不同的编辑器,不同的系统下打开,产生乱码,无法撤回修改,有什么建议?}{}

目前大力建议用户使用 UTF-8 编码


\faq{TeXStudio 里面某些命令明明没有打错,也可以正常编译,为什么编辑框会显示为红的?}{}

TeXStudio 编辑器并没有把所有的命令都写入它的格式文件,即 |cwl| 文件。还有很多时候是语法检查的问题,要么设置相应的语言
包,要么关闭语法检查功能。


\faq{\TeXLive{}怎么手动添加其他宏包?}{}

一般而言 \TeXLive{} 默认把宏包都安装在本地,如果是用户自定义的宏包,只需放置于工作目录。


%\faq{在 macOS 系统下怎么配置中文,很多中文模板用不了?}{}


%\faq{WinEdt 编辑器中如何搜索并高亮出全部的关键词?自己尝试只标记出一个来。}{}


\faq{不同IDE写的源文档相互编译会不会出现不兼容情况?比如用 WinEdt 写的文档,用 xelatex 编译,当文档换到TeXstudio中也用xelatex编译时会不会出现兼容性的错误?}{}

只要文档编码、编译引擎一致,就不会出问题。

%\faq{TeXstudio相比于Winedt有哪些优势?}{}

%\faq{Mactex端打开别模板经常乱码要如何处理?}{}

%\faq{tex2017版的安装 一直安装不了 这个版本和之前的有什么区别吗?改进很多吗?xetex的运行 具体应该怎么运行?尝试过几次,都没有成功}{}

%\faq{Texstudio的菜单栏能否自定义?如何自定义?}{}
