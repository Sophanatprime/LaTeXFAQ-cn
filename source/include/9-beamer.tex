\section{Beamer篇}
%
%
%\begin{faq}{129.隐藏导航栏}
%
%Beamer
%自带的导航符号看起来很不错,但是实际上使用的并不多,为了让文稿的显示面积增加,减少干扰元素,我们可以隐藏下方的导航栏符号,两个方法如下:
%
%\begin{verbatim}
%\setbeamertemplate{navigation symbols}{}
%\beamertemplatenavigationsymbolsempty % both ok
%\end{verbatim}
%
%如果需要去掉下方 title,Author 等信息的话,可以用
%
%\begin{verbatim}
%\setbeamertemplate{footline}
%\end{verbatim}
%\end{faq}
%
%
%\begin{faq}{向 Beamer
%  中添加参考文献}
%
%我们可以使用下面的命令添加参考文献,最好放在 `appendix' 后面。
%
%\begin{verbatim}
%\begin{frame}[allowframebreaks]{References}
%\def\newblock{}
%\bibliographystyle{plain}
%\bibliography{mybib}
%\end{frame}
%\end{verbatim}
%\end{faq}
%
%
%\begin{faq}{每节显示目录}
%
%在我们做一个比较长的报告时,我们可能会想在每一节添加一个目录,让听众清楚内容讲到哪了,我们可以在导言区添加如下的命令。
%
%\begin{verbatim}
%\setbeamerfont{myTOC}{series=\bfseries,size=\Large}
%\AtBeginSection[]{\frame{\frametitle{Outline}%
%\usebeamerfont{myTOC}\tableofcontents[current]}}
%\end{verbatim}
%
%为了得到节的标题信息,我们会在帧与帧之间添加
%`\textbackslash{}section{[}short\_title{]}\{long\_title\}', 其中
%short\_title 是短标题,用于 ``页眉''
%信息(header)显示。如果你不想要显示每帧的页眉信息(header),可以使用下面的命令。
%
%\begin{verbatim}
%\setbeamertemplate{headline}{}
%\end{verbatim}
%\end{faq}
%
%
%\begin{faq}{多栏显示}
%
%有时候我们有图需要并排摆放,一个好方法是使用分栏,尤其是当两个图不同的高度的时候,然后在每一栏插入我们需要的图片。代码如下:
%
%\begin{verbatim}
%\begin{columns}[c] % Columns centered vertically.
%\column{5.5cm}     % Adjust column width to taste.
%\includegraphics ...
%\column{5cm}
%\includegraphics ...
%\end{columns}
%\end{verbatim}
%\end{faq}
%
%
%\begin{faq}{添加 LOGO}
%
%在右下方添加 logo,直接用系统默认的命令就可以。
%
%\begin{verbatim}
%\logo{\includegraphics[width=0.08\textwidth]{logo500}}
%\end{verbatim}
%
%如果需要在右上方添加 logo,可以用 TikZ 命令(需要用到 tikz 宏包)在
%Frametitle 上添加。
%
%\begin{verbatim}
%\addtobeamertemplate{frametitle}{}{%
%\begin{tikzpicture}[remember picture,overlay]
%\node[anchor=north east,yshift=2pt] at (current page.north east) 
%{\includegraphics[width=0.09\textwidth]{logo500}};
%\end{tikzpicture}}
%\end{verbatim}
%\end{faq}
%
%
%\begin{faq}{想在 beamer 中新建一个包含 frame 的环境
%  question,该怎么做?}
%
%直接给代码
%
%\begin{verbatim}
%\newenvironment{question}
%{\begin{frame}[environment=question,fragile]
%\begin{theorem}
%}
%{\end{theorem}
%\end{frame}
%}
%\end{verbatim}
%\end{faq}
%
%
%\begin{faq}{如何在默认模板的基础上,定制自己的beamer模板}
%\end{faq}
%
%
%\begin{faq}{如何更改beamer中logo的位置,在使用default的模板和主题下,使用\cs{logo},发现不能更改logo所在位置}
%\end{faq}
%
%
%
