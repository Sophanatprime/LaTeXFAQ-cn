% !TEX root = ../latex-faq-cn.tex

\section{开发篇-含LaTeX3}

介绍宏开发技巧,宏包和模板类开发的常见问题。


\faq{在阅读已有的宏包或者文类时,遇到未知的命令应如何处理}{}

可以参照胡伟的《LaTeX2e文类和宏包学习手册》中的第四章-命令集注。


\faq{文档类中键值对的实现}{}

使用 xkeyval 宏包。
该宏包主要定义了以下四个命令
\begin{itemize}
  \item Ordinary keys

    最基础的键值对定义方式。说明文档中,这部分主要介绍的是 xkeyval 定义出来的宏的参数和命令。

  \item Command keys

    用于定义存放数据的宏,如美赛模板中运用了该命令来存放参赛队伍编号。

  \item Choice keys

    用于定义一个具有选择性的宏,例如以下宏
\begin{texlist}
  \define@choicekey{HTNotes.cls}{cvcolor}%
  [\val\ar]{green,orange,violet,blue}[blue]{
    \ifcase\ar\relax
      \definecolor{themecolor}{HTML}{92D14F}
      \definecolor{cvtext}{HTML}{92909B}
    \or
      \definecolor{themecolor}{RGB}{230,140,20}
      \definecolor{cvtext}{RGB}{100,100,100}
    \or
      \definecolor{themecolor}{RGB}{178,10,142}
      \definecolor{cvtext}{RGB}{100,100,100}
    \else
      \definecolor{themecolor}{RGB}{0,164,215}
      \definecolor{cvtext}{RGB}{100,100,100}
    \fi
  }
\end{texlist}
  该宏实现的是cvcolor的选择性,可以选择 green, orange, violet, blue 等参数,且 blue 为默认参数。

  \item Boolean keys

    用于定义布尔宏,用于设置开关代码。美赛模板中大部分使用的是该宏。
\end{itemize}

  具体用法请参考其说明文档。

  使用了 xkeyval 后,如何进行参数传递之类的呢?
\begin{texlist}
  \DeclareOptionX*{\PassOptionsToClass{\CurrentOption}{book}}
  \ProcessOptionsX\relax
  \LoadClass[twoside,11pt]{book}
\end{texlist}

  那么,如何定义默认参数呢?
\begin{texlist}
  \ExecuteOptionsX{
    cvcolor = {blue},
    ...
  }
\end{texlist}

  如何设置接口呢?
\begin{texlist}
  \newcommand{\HTset}[1]{\setkeys{HTNotes.cls}{#1}}
\end{texlist}

  如何检测宏是否有内容呢?并依次为基础设置开关命令?
\begin{texlist}
  % 宏定义为
  % \define@cmdkey{HTNotes.cls}[Book@]{Writer}
  \ifdefempty{\Book@Writer}{\@author}{\Book@Writer}
\end{texlist}
  该命令实现的是检测 \verb|\Book@Writer| 宏是否有内容,如果有则使用其存储的内容,无则使用 \verb|\@author| 存储的参数。

  用 xkeyval 提供的键值对实现方式,基本不会出现宏包冲突的问题,因为其前缀是由其特定的包含方式的,前提是选取适当。

  \faq{有没有 cls 文件开发的相关教程文档}{}

  很遗憾,没有。但有两个答案或许能够帮助用户进行 cls 的开发。见以下参考文献。

  \begin{reference}
    \item \href{https://www.zhihu.com/question/27679135/answer/37610214}{开发维护一个 TeX 宏包是怎样的体验?}
    \item \href{https://www.zhihu.com/question/27017364/answer/34932199}{开发一个 LaTeX 宏包需要多少知识?}
  \end{reference}