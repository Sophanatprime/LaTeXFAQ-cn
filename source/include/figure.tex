% Copyright (C) 2018 by latexstudio <http://www.latexstudio.net>
%
% This program is free software: you can redistribute it and/or modify
% it under the terms of the GNU General Public License as published by
% the Free Software Foundation, either version 3 of the License, or
% (at your option) any later version.
%
% This program is distributed in the hope that it will be useful,
% but WITHOUT ANY WARRANTY; without even the implied warranty of
% MERCHANTABILITY or FITNESS FOR A PARTICULAR PURPOSE.  See the
% GNU General Public License for more details.
%
% You should have received a copy of the GNU General Public License
% along with this program.  If not, see <http://www.gnu.org/licenses/>.
%

\section{图片篇}


\faq{LaTeX可以插图哪些类型的图片?}{latex-figure}

我们通常使用LaTeX、PDFTeX、XeTeX编译源文件。各种编译方式下图形格式支持如下
\begin{itemize}
    \item LaTeX直接支持EPS、PS图形文件,间接支持JPEG、PNG等格式
    \item PDFTeX直接支持PNG、PDF、JPEG格式图形文件,间接支持EPS
    \item XeLaTeX直接支持BMP、JPEG、PNG、EPS、PDF图形格式。如果你使用MacOS,那么
    XeLaTeX 还会支持 GIF、PICT、PSD、SGA、TGA、TIFF 等格式 。
\end{itemize}

【注意】在使用PDFLaTeX时,如果要插入EPS,可以先把EPS转化为其他格式(比如PDF、JPEG、PNG、EPS),或者在导言区加载epstopdf,此宏包需要在graphicx宏包之后调用。更改图片格式可以使用ImageMagick或者类似\href{http://www.gaitubao.com}{改图宝}等在线改图软件。
eps 和 pdf 两种格式。eps 是一种在 TeX 中很常用的矢量绘图格式。支持导出
eps 格式的绘图软件包括:MATLAB、Mathematica、GNUPlot、 Asymptote 等。
如果需要使用 pdf 文档中的现成的矢量图,不要使用截屏软件截取,否则
会生成位图,造成失真。可以用 Acrobat 等软件进行提取,剪切。如果使 用
MacOS 系统,可以通过 Skim 阅览器选取,
复制,从剪切板生成笔记的方法导出图像。

\faq{在子文档中想用主文档所在文件夹下的子文件夹内的图片?}{subdoc-figure}

关键在于找到图片,直接暴力使用指定路径的方法,MWE如下。

\begin{verbatim}

main----subfile
     |--figure

main.tex in main folder, figure.png in figure folder, sub.tex in subfile folder.

main.tex:
% !TeX program = pdflatex
\documentclass{article}
\usepackage{graphicx}
\begin{document}
  \include{./subfile/sub1}
\end{document}

sub.tex:
% !TeX root = ../file.tex
\section{test}
hello! \LaTeX{}!
\includegraphics[width=\linewidth]{../figure/figure.png}

\end{verbatim}

但是此种情况有问题,就是不能够使用 \textbackslash{}graphicspath
指定插图路径。这个就留给后来人去解决吧。

\faq{图片的路径如何自动设置,不用正文一个个设置路径?}{figure-path-set}

可以使用指令graphicspath来设置图片路径,如:
\begin{texinlist}
\graphicspath{{./figures/}}
\end{texinlist}

即设定图片路径为当前目录下子文件夹figures。


\faq{图片浮动如何控制?各自参数如何使用?}{figure-control}

插图(figure)、表格(table)等浮动体浮动位置有四个选项可以控制,分别是 h --
here(当前位置), t -- top (页面顶部), b -- bottom(页面底部)和 p --
page(单独一个浮动页)。这四个位置选项的输入顺序是无所谓的,也就是说
{[}htbp{]} 和 {[}btph{]} 的效果是一样的。LaTeX
总是按照h-t-b-p的顺序依次尝试浮动,直到找到合适的位置。LaTeX
标准文档类中对位置参数的默认值是{[}tbp{]},可以通过重定义内部命令
\cs{fps@figure} 和\cs{fps@table} 来修改。

\begin{texinlist}
\makeatletter
\def\fps@figure{htbp}
\def\fps@table{htbp}
\makeatother
\end{texinlist}

LaTeX 放置浮动体时,浮动体不能造成页面溢出(overfull
page),且只能放置于当前页或后面的页面中,浮动体根据其类型必须按源码内出现的顺序出现,也就是说,只有当之前的插图都被处理之后才能对下一幅插图进行处理,那么,只要前面有未处理的插图,当前位置就不会放置插图,一幅不可放置的插图将阻碍其后的图形放置,直到文件结束或出现|\clearpage| 等处理所有未处理浮动体的命令出现之处。

需要说明的是,对于两种浮动体类型,表格的排版和插图的排版是相互独立处理的,未处理的表格不会影响插图的布置。一般来说,给出的参数越多,排版的结果就越好,单个参数选项极容易引发问题,一旦浮动体不适合指定位置,将被搁置并阻碍接下来其他浮动体的处理,一旦被阻塞的浮动体超过LaTeX允许的最大值,还将产生错误。

LaTeX还设定了一些计数器来限制页面上浮动体的数量,见表 \ref{figure-counter}。

\begin{table}[ht!]
  \centering
  \begin{longtable}{|c|c|}
    \hline
    计数器 & 含义 \\
    \hline
    \endhead
    dbltopnumber & twocolumn 模式下可以位于页面顶部的浮动体最大数目(缺省为2) \\
    \hline
    topnumber & 可以位于页面顶部的浮动体最大数目(缺省为2) \\
    \hline
    bottomnumber & 可以位于页面底部的浮动体最大数目(缺省为1)\\
    \hline
    totalnumber & 可以位于文本页中的浮动体最大数目(缺省为3) \\
    \hline
  \end{longtable}
  \caption{浮动体计数器含义}
  \label{figure-counter}
\end{table}

LaTeX 还设定了一些比例参数控制浮动体的放置,见表 \ref{figure-params}。

\begin{table}[ht!]
  \centering
  \begin{longtable}{|c|c|}
    \hline
    参数 & 含义 \\
    \hline
    \endhead
    |\textfraction| & 文本页上文本最小比例(默认0.2) \\
    \hline
    |\topfraction| & 页面顶部浮动体高度比例(默认0.7) \\
    \hline
    |\bottomfraction| & 页面底部浮动体高度比例(默认0.3) \\
    \hline
    |\floatpagefraction| & 浮动页浮动体高度比例(默认0.5) \\
    \hline
    |\dbltopfraction| & twocolumn 模式下页面顶部浮动体高度比例(默认0.7)\\
    \hline
    |\dblfloatpagefraction| & twocolumn 模式下浮动页浮动体高度比例(默认0.5) \\
    \hline
  \end{longtable}
  \caption{浮动体参数含义}
  \label{figure-params}
\end{table}


这些计数器和比例值可以通过 \cs{setcounter} 和\cs{renewcommand}
分别进行调整。但调整时应特别小心,不适当的比例值会导致非常糟糕的排版或大量未处理的浮动体。如果只是需要LaTeX在处理某一浮动体时忽略以上这些限制条件,可以在浮动体位置选项参数中加!即可。注意,!
对 浮动页限制条件的忽略无效。

\begin{texinlist}
\begin{table}[!hbt]
  the contents of the table ...
\end{table}
\end{texinlist}


\faq{图文混排用什么方法实现?}{figure-text-inline}

大概有好几个宏包:picinpar、wrapfig,以及过时了的 picins
宏包。但是都有或多或少的问题,都不能够做得比较智能。等着后来人的修订以及更好的实现方式吧。

\begin{itemize}
  \item wrapfig 用法
  \begin{texinlist}
\begin{wrapfigure}{行数}{位置}{超出长度}{宽度}
  <图形>
\end{wrapfigure}
  \end{texinlist}
  \begin{itemize}
    \item 行数
    是指图形高度所占的文本行的数目,如果不给出此选项, wrapfig 会自动计算。
    \item 位置
    是指图形相对于文本的位置,须给定下面四项的一个。 
    \begin{description}
      \item[r,R] 表示图形位于文本的左边。
      \item[l,L] 表示图形位于文本的右边。
      \item[i,R] 表示图形位于页面靠里的一边(用在双面格式里)。
      \item[o,O] 表示图形位于页面靠外的一边。
    \end{description}
    \item 超出长度
    是指图形超出文本边界的长度,缺省为 0pt。 
    \item 宽度
    指图形的宽度。 wrapfig 会自动计算 图形的高度。不过,我们也可设定图形的高度,具体可见 wrapfig.sty 内 的说明。
  \end{itemize}
  \item picinpar 用法 \\
  picinpar 宏包定义了一个基本的环境 window,还有两个变体  figwindow 和 tabwindow。允许在文本段落中打开一个``窗口 '', 在其中放入图形、文字和表格等。这里我们主要讨论将图形放入文本段落 的用法,其它的用法可参考 picinpar 的说明。  
  \begin{texinlist}
\begin{window} [行数,对齐方式,内容,内容说明]\end{window}
\begin{figwindow} [行数,对齐方式,图形,标题]\end{figwindow}
  \end{texinlist}
  \begin{itemize}
    \item 行数是指“窗口”开始前的行数。 
    \item 对齐方式是指在段落中“窗口'“的对齐方式。缺省为 l, 即左对齐。 另外两种是 c :居中和 r :右对齐 。
    \item 第三个参数是出现在“窗口”中的“内容”,这在 figwindow 中就是 要插入的图形。第四个参数则是对``窗口''内容的说明性文字,这在  figwindow 中就是图形的标题。
  \end{itemize}
\end{itemize}


\faq{并列插图如何进行排版}{col-figure-install}

并列插图有3种情况:

\begin{itemize}
    \item 并排摆放,各有标题。\\
    可以在figure环境中使用两个minipage环境,每个里面插入一幅插图。
    
    \begin{texinlist}
    \begin{figure}[htbp]
    \centering
    \begin{minipage}{60pt}
    \centering \includegraphics[scale=0.4]{leftfigure.png} \caption{左边的图片}
    \end{minipage}
    \hspace{10pt}%用来调整图片中间的间距
    \begin{minipage}{60pt}
    \centering
    \includegraphics[scale=0.4]{rightfigure.png} \caption{右边的图片}
    \end{minipage}
    \end{figure}
    \end{texinlist}
    \item 并排摆放,共享标题。\\
    通过使用两个 \cs{includegraphics} 命令

   \begin{texinlist}
\begin{figure}[htbp]
\centering
\includegraphics{leftfig.png}
\includegraphics{rightfig.png}
\caption{总标题}
\end{figure}
    \end{texinlist}
    
    \item 并排摆放,共享标题,并且有各自的子标题 \\
    如果想要两幅并排的图片共享一个标题,并且各有自己的子标题, 可以使用Steven D. Cochran 开发的 subfig 宏包。它提供的 \cs{subfloat} 命令,并且总图和子图可以分别有标题和引用。
    
    \begin{texinlist}
\begin{figure}[htbp]
  \centering
  \subfloat[左边图片的标题]{
    \label{fig:subfig_a}
    \includegraphics[scale=0.4]{leftfig.png}
  }
  \hspace{10pt}% 用来调整两图中间的间距
  \subfloat[右边图片的标题]{
    \label{fig:subfig_b}
    \includegraphics[scale=0.4]{rightfig.png}
  }
  \caption{总标题}
  \label{fig:subfig}
\end{figure}
    \end{texinlist}

    此外,如果是并列的是两个有各自标题的插图,可以使用floatrow系列浮动体宏包,该宏包提供的floatrow环境可以并列图表等浮动体。
    
\end{itemize}


\faq{并列子图如何进行排版}{col-sub-figure}

并列子图可以看看subfigure,subfloat、subcaption等宏包。


\faq{如果想让图片的题注在图片右侧,应该怎么做}{figure-right}

可以利用盒子来实现这个功能。下面给出一个例子

\begin{verbatim}
\documentclass{article}
\usepackage{graphicx}
\begin{document}
    \begin{figure}
    \centering
    \includegraphics[width=0.45\linewidth]{figure.png}
    \parbox[b]{0.45\linewidth}{\caption{the content of caption}}
  \end{figure}
\end{document}
\end{verbatim}

若要让题注在图片左侧,只需将 \textbackslash{}parbox 那段代码移到
\textbackslash{}includegraphics 之前。


\faq{在插图较多,文字较少的情况下,正文会产生较多空白,或者单个图片占一页的情况,如何处理?}{ins-fig-or-text}

尽量避免这样的行文方式,比如可以将图片以附录形式集中排版。单个图片占一页在绝大多数情况下都不需要处理,浮动体页是很常见的形式。只有当图片恰好出现在一章的结尾,正文正好排满一页后换页,而图表本身尺寸又不大的时候,图表以浮动页排版方式排在页面正中有些突兀,这时可以通过浮动选项设置{[}!ht{]}要求其在页面顶部排版,并忽略latex从美学角度出发对浮动体做出的一些限制。


\faq{在双栏文档中,如何插入单栏图片,表格?}{col-ins-fig-tab}

要看双栏文档是如何实现的。若双栏文档的实现方式是文档类的 twocolumn
选项实现的,那么用带*形式的浮动体环境替代原浮动体环境即可,这时的浮动选项只有tp有效;若双栏文档是以
multicol 宏包的 multicols 环境实现的,那么,在 multicols
环境内不支持浮动体,当需要插入单栏图片表格时,可结束multicols环境,待插入图片、表格后,重新开启multicols
环境。


\faq{不想让图片浮动,又想使用caption,如何二者兼得?}{fig-cap}

caption宏包提供了一个
\cs{captionof}
命令,可以在浮动体环境外使用,命令的语法格式是:\cs{captionof}\oarg{float
type}\oarg{list entry}\marg{heading},举例如下:

\begin{verbatim}
\begin{center}
\includegraphics{example-image.pdf}
\captionof{figure}{the example}
\end{center}
\end{verbatim}

不过非常不建议使用这种方式,浮动体是一种很好的处理图表的方式。


\faq{有没有办法把图片固定在某位置}{fig-on-this-pos}

不使用浮动体就会在你指定的位置出现了,但是非常非常不可取,一般不建议这么搞。


%\faq{如何可以写一段话,放张图片,再写一段话,再放图片。}{}


%\faq{如何在一张图片上再叠放另外一张图片?如下图,在图中小孩的白板上分别加一个对号和叉号。}{}

% \includegraphics{https://qqadapt.qpic.cn/txdocpic/0/cb8fb575407168ba7c289778e7f7c526/0}
