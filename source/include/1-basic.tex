\section{背景知识与基本概念}

\begin{faq}{什么是 \TeX{}?}
\TeX{} 是由著名计算机科学家 Donald~E. Knuth(高德纳)发明的排版系统。他在《The \TeX book》一书的
前言中曾提到:“(\TeX{})旨在创造美丽的书籍,尤其是那些包含很多数学公式的书。”
\footnote{原文如下:“This is a handbook about \TeX{}, a new typesetting system intended for the
creation of beautiful books---and especially for books that contain a lot of mathematics.”}

1976 年,Knuth 出版鸿篇巨著《The Art of Computer Programming》第二卷的第二版,但当时所用的照排技术
却令他非常失望。作为斯坦福大学计算机科学系的教授,Knuth 决定自己开发一套高质量的排版系统。1978 年,
他开发出了 \TeX{} 的第一个版本;随后,又在 1982 年推出了 \TeX{} 的第二个版本(\TeX 82),也就是
人们今天所用 \TeX{} 的基础。Knuth 将 \TeX{} 的源代码无偿发布在公有领域
\footnote{\TeX{} 使用的许可证为 \href{https://www.ctan.org/license/knuth}{Knuth License}。},
这使得他人可以进一步完善这一系统,并增加新的功能。

在今天,\TeX{} 既可以指 Knuth 发明的这一套排版系统,也可以指相应的排版语言,有时候也指将其打包、
整理以方便用户使用的软件套装(发行版)。

\begin{reference}
  \item https://texfaq.org/FAQ-whatTeX
  \item \TeX{}, Wikipedia, The Free Encyclopedia, \url{https://en.wikipedia.org/wiki/TeX}
\end{reference}

% Knuth 是美国加州斯坦福大学计算机编程专业的名誉教授,他在 1978 年开发了第一个版本的 \TeX{} 用来处理
% 他的《计算机编程艺术》的修订。这个想法特别受欢迎,所以1982年他推出了 \TeX{} 的第二个版本,也就是
% 人们今天所用的 \TeX{} 的基础。
%
% Knuth 开发了一套“文学编程”系统来编写 \TeX{},他还提供了免费的 \TeX{} 资源,以及可以将网络
% 资源转化为可以编译或者打印的东西的工具;Knuth 做了什么,对人们来说从来都不是什么神秘的事情。
% \TeX{} 以及它的文档都是高度可移植的。
%
% 对于感兴趣的程序员来说,\TeX{} 有其迷人之处:没有什么能比得上一个人可以构建这样一个程序,至今它的
% 持续时间比大多数的程序都好,而且已经被移植到了许多不同的计算机构架和操作系统中许多现代编程所追求
% 的属性。经过处理的“可读”的 \TeX{} 程序源代码可以在 TDS structured 版本中找到。

% 来源 https://texfaq.org/FAQ-whatTeX
\end{faq}

\begin{faq}{什么是 \LaTeX{}?}
\end{faq}

\begin{faq}{\TeX{} 中常见术语的解释}
\textbf{引擎}
  
\LaTeX/\TeX{}解析引擎,其实就是一个编译器,它输入一个\verb|.tex|文件作为输入,根据源文件的内容送入解析引擎和渲染引擎进行处理,并将排版的成果——文档编译输出,\LaTeX/\TeX{}的解析引擎目前有pdflatex、xelatex、lualatex等,它们都可以输出pdf文档文件(部分解析器可以输出dvi文件),用于在多平台进行分发甚至打印出版。

\textbf{格式}

\TeX{} 是存在各种不同的封装格式的,比如原生的 \TeX{} 或者 \LaTeX{},我们所使用的 \LaTeX{} 只是\TeX{} 封装格式的其中一种,是目前流行的封装规范。

\textbf{发行版}

\LaTeX/\TeX{}都包含了成千上万个宏包,甚至有可能我们需要安装新的宏包,除了手动安装外,最好的方式就是利用发行版的宏包管理器,所谓发行版就是把\LaTeX/\TeX{}的相关组件打包,形成一个独立完善的\LaTeX/\TeX{}系统,目前流行的发行版有MiKTeX、proTeXt 以及TeXLive。
\end{faq}


\begin{faq}{}

\end{faq}

\begin{faq}{\LaTeX{} 源文件有什么要求?}
LaTeX的源文件是 *.tex文件,是指latex编译器处理输入文件的源码,latex编译器会对输入文件进行解析,构造解析树,进行渲染,然后输出处理后的文档,完成一次编译过程,由于LaTeX解析器可能对中文文件名处理存在兼容性问题,不建议将LaTeX的源文件的文件名设置为中文。
\end{faq}
