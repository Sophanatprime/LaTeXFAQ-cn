% Copyright (C) 2018 by latexstudio <http://www.latexstudio.net>
%
% This program is free software: you can redistribute it and/or modify
% it under the terms of the GNU General Public License as published by
% the Free Software Foundation, either version 3 of the License, or
% (at your option) any later version.
%
% This program is distributed in the hope that it will be useful,
% but WITHOUT ANY WARRANTY; without even the implied warranty of
% MERCHANTABILITY or FITNESS FOR A PARTICULAR PURPOSE.  See the
% GNU General Public License for more details.
%
% You should have received a copy of the GNU General Public License
% along with this program.  If not, see <http://www.gnu.org/licenses/>.
%

\section{最常出现的问题}
\label{sec:starter}

\faq{论文 / 比赛的 deadline 要到了,如何在一天 / 两天 / 三天之内入门 \LaTeX{}?}{get-started}

非常遗憾,这几乎是不可能完成的任务。在时间紧张、压力巨大的情形下,入门 \LaTeX{} 对您来说没有意义。
作为排版工具,\LaTeX{} 实现的效果远没有文章的内容重要,所以请不要在这种情况下把您的精力投入在学习
\LaTeX{} 上。

通常来说,我们建议您至少通过三个月的时间来入门 \LaTeX{},并在之后的工作、学习中不断深入理解、
积累经验。

如果确有必要在短时间内掌握 \LaTeX{} 的使用方法,请联系靠谱且有经验的人。注:很多“学长”、“老师”都是
不靠谱的,所以这一点实际上很难办到。


\faq{如何“安装 \LaTeX{}”?}{install-latex}

如果您也有这一问题,首先需要澄清一些概念,见~\faqref{tex-terms}。简短来说,\LaTeX{} 本身是一种标记
语言,而不是 Microsoft Word 一样现成的软件。因此,社区将相关的支持文件、可执行程序、文档等打包在了
一起,形成了可供用户下载、安装的发行版(distribution)。一般而言,“安装 \LaTeX{}”指的就是安装
发行版。

目前,根据平台不同,可供使用的主流发行版有以下这些:

\begin{itemize}
  \item \TeXLive{},见~\faqref{install-texlive}
  \item \MiKTeX{},见~\faqref{install-miktex}
  \item \MacTeX{},适用于 macOS,见~\faqref{install-mactex}
  % TODO: 更多的发行版介绍
  \item 【有待整理】
\end{itemize}


\faq{我的 \CTeX{} 为什么……}{why-ctex}

您在这里提到的“\CTeX{}”,指的很可能是由中国 \TeX{} 社区(即 \CTeX{} 社区)所发布的、以 \MiKTeX{}
为基础的一个发行版,全称为“\href{http://www.ctex.org/CTeX}{\CTeX{} 套装}”。这一发行版目前已停止
维护,所以除非必要,请不要再使用。\TeXLive{} 以及 \MiKTeX{} 都是可靠的替代方案。

\CTeX{} 社区另发布了一个同样称为 \href{https://www.ctan.org/pkg/ctex}{\CTeX{} 的宏集}(宏包),
这是目前在 \LaTeX{} 中使用中文排版的推荐方案。如果您需要获取相关信息,请查阅其文档。\CTeX{} 宏集
的有关问题,在本文之后也有涉及。
