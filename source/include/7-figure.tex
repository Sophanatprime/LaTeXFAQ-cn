\section{图片篇}
%
%
%\begin{faq}{LaTeX可以插图哪些类型的图片?}
%
%我们通常使用LaTeX、PDFTeX、XeTeX编译源文件。各种编译方式下图形格式支持如下
%* LaTeX直接支持EPS、PS图形文件,间接支持JPEG、PNG等格式; *
%PDFTeX直接支持PNG、PDF、JPEG格式图形文件,间接支持EPS; *
%XeLaTeX直接支持BMP、JPEG、PNG、EPS、PDF图形格式. 如果你使用MacOS,那么
%XeLaTeX 还会支持 GIF、PICT、PSD、SGA、TGA、TIFF 等格式 .
%
%【注意】在使用PDFLaTeX时,如果要插入EPS,可以先把EPS转化为其他格式(比如PDF、JPEG、PNG、EPS),或者在导言区加载epstopdf,此宏包需要在graphicx宏包之后调用。更改图片格式可以使用ImageMagick或者类似\href{http://www.gaitubao.com}{改图宝}等在线改图软件。
%eps 和 pdf 两种格式。eps 是一种在 TeX 中很常用的矢量绘图格式。支持导出
%eps 格式的绘图软件包括:MATLAB、Mathematica、GNUPlot、 Asymptote 等。
%如果需要使用 pdf 文档中的现成的矢量图,不要使用截屏软件截取,否则
%会生成位图,造成失真。可以用 Acrobat 等软件进行提取,剪切。如果使 用
%MacOS 系统,可以通过 Skim 阅览器选取,
%复制,从剪切板生成笔记的方法导出图像。
%\end{faq}
%
%
%\begin{faq}{图片的路径如何自动设置,不用正文一个个设置路径?}
%
%可以使用指令graphicspath来设置图片路径,如:
%\begin{verbatim}
%\graphicspath{{./figures/}}
%\end{verbatim}
%
%即设定图片路径为当前目录下子文件夹figures。
%\end{faq}
%
%
%\begin{faq}{115.在子文档中想用主文档所在文件夹下的子文件夹内的图片?}
%
%关键在于找到图片,直接暴力使用指定路径的方法,MWE如下。
%
%\begin{verbatim}
%main----subfile
%|--figure
%
%main.tex in main folder, figure.png in figure folder, sub.tex in subfile folder.
%
%main.tex:
%% !TeX program = pdflatex
%\documentclass{article}
%\usepackage{graphicx}
%\begin{document}
%\include{./subfile/sub1}
%\end{document}
%
%sub.tex:
%% !TeX root = ../file.tex
%\section{test}
%hello! \LaTeX{}!
%\includegraphics[width=\linewidth]{../figure/figure.png}
%\end{verbatim}
%
%但是此种情况有问题,就是不能够使用 \textbackslash{}graphicspath
%指定插图路径。这个就留给后来人去解决吧。
%\end{faq}
%
%
%\begin{faq}{图片浮动如何控制?各自参数如何使用?}
%
%插图(figure)、表格(table)等浮动体浮动位置有四个选项可以控制,分别是 h --
%here(当前位置), t -- top (页面顶部), b -- bottom(页面底部)和 p --
%page(单独一个浮动页)。这四个位置选项的输入顺序是无所谓的,也就是说
%{[}htbp{]} 和 {[}btph{]} 的效果是一样的。LaTeX
%总是按照h-t-b-p的顺序依次尝试浮动,直到找到合适的位置。LaTeX
%标准文档类中对位置参数的默认值是{[}tbp{]},可以通过重定义内部命令
%\cs{fps@figure} 和\cs{fps@table} 来修改。
%
%\begin{verbatim}
%\makeatletter
%\def\fps@figure{htbp}
%\def\fps@table{htbp}
%\makeatother
%\end{verbatim}
%
%LaTeX 放置浮动体时,浮动体不能造成页面溢出(overfull
%page),且只能放置于当前页或后面的页面中,浮动体根据其类型必须按源码内出现的顺序出现,也就是说,只有当之前的插图都被处理之后才能对下一幅插图进行处理,那么,只要前面有未处理的插图,当前位置就不会放置插图,一幅不可放置的插图将阻碍其后的图形放置,直到文件结束或出现\clearpage
% 等处理所有未处理浮动体的命令出现之处。
%
%需要说明的是,对于两种浮动体类型,表格的排版和插图的排版是相互独立处理的,未处理的表格不会影响插图的布置。一般来说,给出的参数越多,排版的结果就越好,单个参数选项极容易引发问题,一旦浮动体不适合指定位置,将被搁置并阻碍接下来其他浮动体的处理,一旦被阻塞的浮动体超过LaTeX允许的最大值,还将产生错误。
%
%LaTeX还设定了一些计数器来限制页面上浮动体的数量,这些包括:
%dbltopnumber\textbar{}twocolumn
%模式下可以位于页面顶部的浮动体最大数目(缺省为2)\textbar{}
%:----\textbar{}:----\textbar{} topnumber
%\textbar{}可以位于页面顶部的浮动体最大数目(缺省为2)\textbar{}
%bottomnumber\textbar{}可以位于页面底部的浮动体最大数目(缺省为1)\textbar{}
%totalnumber\textbar{}可以位于文本页中的浮动体最大数目(缺省为3)\textbar{}
%
%LaTeX 还设定了一些比例参数控制浮动体的放置,包括
%\textfraction\textbar{}文本页上文本最小比例(默认0.2)\textbar{}
%:----\textbar{}:----\textbar{}
%\topfraction\textbar{}页面顶部浮动体高度比例(默认0.7)\textbar{}
%\bottomfraction\textbar{}页面底部浮动体高度比例(默认0.3)\textbar{}
%\floatpagefraction\textbar{}浮动页浮动体高度比例(默认0.5)\textbar{}
%\dbltopfraction\textbar{}twocolumn
%模式下页面顶部浮动体高度比例(默认0.7)\textbar{}
%\dblfloatpagefraction\textbar{}twocolumn
%模式下浮动页浮动体高度比例(默认0.5)\textbar{}
%
%这些计数器和比例值可以通过 \cs{setcounter} 和\cs{renewcommand}
%分别进行调整。但调整时应特别小心,不适当的比例值会导致非常糟糕的排版或大量未处理的浮动体。如果只是需要LaTeX在处理某一浮动体时忽略以上这些限制条件,可以在浮动体位置选项参数中加!即可。注意,!
%对 浮动页限制条件的忽略无效。
%
%\begin{verbatim}
%\begin{table}[!hbt]
%the contents of the table ...
%\end{table}
%\end{verbatim}
%\end{faq}
%
%
%\begin{faq}{图文混排用什么方法实现?}
%
%大概有好几个宏包:picinpar、wrapfig,以及过时了的 picins
%宏包。但是都有或多或少的问题,都不能够做得比较智能。等着后来人的修订以及更好的实现方式吧。
%* wrapfig 用法
%
%\begin{verbatim}
%\begin{wrapfigure}{行数}{位置}{超出长度}{宽度}
%<图形>
%\end{wrapfigure}
%\end{verbatim}
%
%1.行数 是指图形高度所占的文本行的数目,如果不给出此选项, wrapfig
%会自动计算。 2.位置 是指图形相对于文本的位置,须给定下面四项的一个。 r,R
%表示图形位于文本的左边。 l,L 表示图形位于文本的右边。 i,R
%表示图形位于页面靠里的一边(用在双面格式里)。 o,O
%表示图形位于页面靠外的一边。 3.超出长度
%是指图形超出文本边界的长度,缺省为 0pt。 4.宽度 指图形的宽度。 wrapfig
%会自动计算 图形的高度。不过,我们也可设定图形的高度,具体可见
%wrapfig.sty 内 的说明。 * picinpar 用法
%
%picinpar 宏包定义了一个基本的环境 window,还有两个变体 figwindow 和
%tabwindow。允许在文本段落中打开一个\texttt{窗口\ \textquotesingle{}\textquotesingle{},\ 
%在其中放入图形、文字和表格等。这里我们主要讨论将图形放入文本段落\ 的用法,其它的用法可参考\ picinpar\ 
%的说明。\ \textasciigrave{}\textasciigrave{}\textasciigrave{}\ \textbackslash{}begin\{window\}\ 
%{[}行数,对齐方式,内容,内容说明{]}\textbackslash{}end\{window\}\ 
%\textbackslash{}begin\{figwindow\}\ 
%{[}行数,对齐方式,图形,标题{]}\textbackslash{}end\{figwindow\}\ 
%\textasciigrave{}\textasciigrave{}\textasciigrave{}\ **\ **1.是指“窗口”开始前的行数。\ \ 
%2.对齐方式是指在段落中“窗口\textquotesingle{}“的对齐方式。缺省为\ l,\ 即左对齐。\ 另外两种是\ c\ 
%:居中和\ r\ :右对齐。\ \ 3.第三个参数是出现在“窗口”中的“内容”,这在\ figwindow\ 中就是\ 
%要插入的图形。第四个参数则是对}窗口''内容的说明性文字,这在
%figwindow 中就是图形的标题。
%\end{faq}
%
%
%\begin{faq}{并列插图如何进行排版}
%
%并列插图有3种情况: * 并排摆放,各有标题。
%
%可以在figure环境中使用两个minipage环境,每个里面插入一幅插图。
%
%\begin{verbatim}
%\begin{figure}[htbp]
%\centering
%\begin{minipage}{60pt}
%\centering \includegraphics[scale=0.4]{leftfigure.png} \caption{左边的图片}
%\end{minipage}
%\hspace{10pt}%用来调整图片中间的间距
%\begin{minipage}{60pt}
%\centering
%\includegraphics[scale=0.4]{rightfigure.png} \caption{右边的图片}
%\end{minipage}
%\end{figure}
%\end{verbatim}
%
%\begin{itemize}
%  
%  \item
%  并排摆放,共享标题
%\end{itemize}
%
%通过使用两个 \cs{includegraphics} 命令
%
%\begin{verbatim}
%\begin{figure}[htbp]
%\centering
%\includegraphics{leftfig.png}
%\includegraphics{rightfig.png}
%\caption{总标题}
%\end{figure}
%\end{verbatim}
%
%\begin{itemize}
%  
%  \item
%  并排摆放,共享标题,并且有各自的子标题
%\end{itemize}
%
%如果想要两幅并排的图片共享一个标题,并且各有自己的子标题, 可以使用
%Steven D. Cochran 开发的 subfig 宏包。它提供的 \cs{subfloat} 命
%令,并且总图和子图可以分别有标题和引用。
%
%\begin{verbatim}
%\begin{figure}[htbp]
%\centering
%\subfloat[左边图片的标题]{
%\label{fig:subfig_a} \includegraphics[scale=0.4]{leftfig.png}
%}
%\hspace{10pt}% 用来调整两图中间的间距
%\subfloat[右边图片的标题]{
%\label{fig:subfig_b}
%\includegraphics[scale=0.4]{rightfig.png}
%} \caption{总标题} \label{fig:subfig} \end{figure}
%\end{verbatim}
%
%此外,如果是并列的是两个有各自标题的插图,可以使用floatrow系列浮动体宏包,该宏包提供的floatrow环境可以并列图表等浮动体。
%\end{faq}
%
%
%\begin{faq}{并列子图如何进行排版}
%
%并列子图可以看看subfigure,subfloat、subcaption等宏包。
%\end{faq}
%
%
%\begin{faq}{如果想让图片的题注在图片右侧,应该怎么做}
%
%可以利用盒子来实现这个功能。下面给出一个例子
%
%\begin{verbatim}
%\documentclass{article}
%\usepackage{graphicx}
%\begin{document}
%\begin{figure}
%\centering
%\includegraphics[width=0.45\linewidth]{figure.png}
%\parbox[b]{0.45\linewidth}{\caption{the content of caption}}
%\end{figure}
%\end{document}
%\end{verbatim}
%
%若要让题注在图片左侧,只需将 \textbackslash{}parbox 那段代码移到
%\textbackslash{}includegraphics 之前。
%\end{faq}
%
%
%\begin{faq}{在插图较多,文字较少的情况下,正文会产生较多空白,或者单个图片占一页的情况,如何处理?}
%
%尽量避免这样的行文方式,比如可以将图片以附录形式集中排版。单个图片占一页在绝大多数情况下都不需要处理,浮动体页是很常见的形式。只有当图片恰好出现在一章的结尾,正文正好排满一页后换页,而图表本身尺寸又不大的时候,图表以浮动页排版方式排在页面正中有些突兀,这时可以通过浮动选项设置{[}!ht{]}要求其在页面顶部排版,并忽略latex从美学角度出发对浮动体做出的一些限制。
%\end{faq}
%
%
%\begin{faq}{在双栏文档中,如何插入单栏图片,表格?}
%
%要看双栏文档是如何实现的。若双栏文档的实现方式是文档类的 twocolumn
%选项实现的,那么用带*形式的浮动体环境替代原浮动体环境即可,这时的浮动选项只有tp有效;若双栏文档是以
%multicol 宏包的 multicols 环境实现的,那么,在 multicols
%环境内不支持浮动体,当需要插入单栏图片表格时,可结束multicols环境,待插入图片、表格后,重新开启multicols
%环境。
%\end{faq}
%
%
%\begin{faq}{不想让图片浮动,又想使用caption,如何二者兼得?}
%
%caption宏包提供了一个
%\cs{captionof}
%命令,可以在浮动体环境外使用,命令的语法格式是:\cs{captionof}\oarg{float
%  type}\oarg{list entry}\marg{heading},举例如下:
%
%\begin{verbatim}
%\begin{center}
%\includegraphics{example-image.pdf}
%\captionof{figure}{the example}
%\end{center}
%\end{verbatim}
%
%不过非常不建议使用这种方式,浮动体是一种很好的处理图表的方式。
%\end{faq}
%
%
%\begin{faq}{有没有办法把图片固定在某位置}
%
%不使用浮动体就会在你指定的位置出现了,但是非常非常不可取,一般不建议这么搞。
%\end{faq}
%
%
%\begin{faq}{如何可以写一段话,放张图片,再写一段话,再放图片。}
%\end{faq}
%
%
%\begin{faq}{如何在一张图片上再叠放另外一张图片?如下图,在图中小孩的白板上分别加一个对号和叉号。}
%
%% \includegraphics{https://qqadapt.qpic.cn/txdocpic/0/cb8fb575407168ba7c289778e7f7c526/0}
%\end{faq}
%
%
%
