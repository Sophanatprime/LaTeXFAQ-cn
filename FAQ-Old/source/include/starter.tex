% !TEX root = ../latex-faq-cn.tex

\section{最常出现的问题}
\label{sec:starter}

\faq{论文 / 比赛的 deadline 要到了,如何在一天 / 两天 / 三天之内入门 \LaTeX{}?}%
  {get-started-in-a-short-time}

非常遗憾,这几乎是不可能完成的任务。在时间紧张、压力巨大的情形下,入门 \LaTeX{} 对您来说没有意义。
作为排版工具,\LaTeX{} 实现的效果远没有文章的内容重要,所以请不要在这种情况下把您的精力投入在学习
\LaTeX{} 上。

通常来说,我们建议您至少通过三个月的时间来学习 \LaTeX{},并在之后的工作、学习中不断深入理解、
积累经验。

如果确有必要在短时间内掌握 \LaTeX{} 的使用方法,请联系靠谱且有经验的人。

\begin{note}
  很多“学长”、“老师”都是不靠谱的。
\end{note}


\faq{我是新手,学习 \LaTeX{} 应该怎么入门?}{get-started}

首先,您需要确保自己有较为充裕的时间。接下来,我们推荐您阅读一些 \LaTeX{} 入门材料,比较知名的有
\href{http://mirrors.ctan.org/info/lshort/chinese/lshort-zh-cn.pdf}{《一份不太简短的 \LaTeXe{} 介绍》}
(\href{http://mirrors.ctan.org/info/lshort/english/lshort.pdf}{\book{The Not So Short In­tro­duc­tion to \LaTeXe{}}}),
即 \pkg{lshort}。此外,您还需要在电脑上安装一套 \TeX{} 发行版(见\faqref{install-latex}),或使用
在线服务,如 \href{https://www.overleaf.com/}{Overleaf}。
原在线服务 \href{https://www.sharelatex.com}{ShareLaTeX} 已与 \href{https://www.overleaf.com/}{Overleaf} 合并为 \href{https://v2.overleaf.com}{Overleaf v2 平台} 。

进一步的学习还需要阅读更多的书籍和文档,例如刘海洋《\LaTeX{} 入门》~\footnote{如作者所言,“虽然书名
是《入门》,不过其实我并非完全是按新手入门的节奏在写,除了第一章是零基础入门,后面的内容编排并不
完全是从基础到深入的顺序,而是按主题划分;实际内容上也包含一些比较专门的东西”,这本书内容丰富、
讲解深入,但对新手未必友好;而且需注意,关于中文处理的部分可能略显过时。}。此外,善用
搜索引擎~\footnote{对于技术方面的问题,我们不推荐百度。谷歌在大多数时候都可以获得更好的搜索结果;
当然,如果您身处大陆地区,必应很可能是更好的选择。} 也是必备技能。最后,我们强烈建议您提高英语水平,
因为 \LaTeX{} 的相关材料,除了与中文排版相关的内容外,几乎都使用英文撰写。

\begin{reference}
  \item 刘海洋.自学 \LaTeX{} 可以读什么书入门?
    \url{https://www.zhihu.com/question/26645810/answer/33515971}
\end{reference}


\faq{如何“安装 \LaTeX{}”?}{install-latex}

回答这一问题,首先需要澄清一些概念,见\faqref{tex-related-things}。简短来说,\LaTeX{} 本身是
一种标记语言,而不是 Microsoft Word 一样现成的软件。因此,社区将相关的支持文件、可执行程序、文档等
打包在了一起,形成了可供用户下载、安装的\strong{发行版(distribution)}。一般而言,“安装 \LaTeX{}”
指的就是安装发行版。

目前,可供使用的主流发行版有以下这些:

\begin{itemize}
  \item \TeXLive{}
  \item \MacTeX{},它实际上是 \TeXLive{} 的再次封装
  \item \MiKTeX{}
\end{itemize}
% 见\faqref{install-texlive}
% \item \MiKTeX{},见\faqref{install-miktex}
% \item \MacTeX{},适用于 macOS,见\faqref{install-mactex}
% \item pro\TeX{}t,见\faqref{download-protext}
% % TODO: 更多的发行版介绍
% \item 【有待整理】


\faq{我应该在哪里编写 \LaTeX{} 文档?}{where-to-write-latex}

\LaTeX{} 文档是纯文本文件,因此原则上可以在任意的\strong{文本编辑器}中编写。支持 \LaTeX{} 的常见
编辑器有以下这些:

\begin{itemize}
  \item TeXworks、TeXstudio、Texpad、WinEdt 等,它们是专门为编写 \LaTeX{} 设计的;
  \item Visual Studio Code、Sublime Text、Atom 等,它们是基于 Web 技术的现代编辑器,依靠插件支持
    \LaTeX{} 语言;
  \item Vim、Emacs,它们是具有强大可扩展性、可定制性但学习曲线也很陡峭的编程利器,
    同样依靠插件支持。
\end{itemize}

\begin{note}
  由于 Windows 系统中记事本(Notepad)的编码问题,请不要在其中编辑任何含有非 ASCII 字符的文件。
\end{note}


\faq{我的 \CTeX{} 为什么……}{why-ctex}

您在这里提到的“\CTeX{}”,指的很可能是由中国 \TeX{} 社区(即 \CTeX{} 社区)所发布的、以 \MiKTeX{}
为基础的一个发行版,全称为 \href{http://www.ctex.org/CTeX}{\CTeX{} 套装}。这一发行版目前已\strong{停止
维护},所以除非必要,请不要再使用。\TeXLive{} 以及 \MiKTeX{} 都是可靠的替代方案。

\CTeX{} 社区另发布了一个同名的 
\href{https://www.ctan.org/pkg/ctex}{\CTeX{} 宏集},这是目前在 \LaTeX{} 中使用中文排版的推荐方案。
如果您需要获取相关信息,请查阅其文档。\CTeX{} 宏集的有关问题,在本文之后也有涉及。


\faq{那些含有 \TeX{}、\LaTeX{} 的都是什么东西?}{tex-related-things}

\TeX{} 本身是由 Knuth 发明的一套排版系统,同时也指相应的排版语言(见\faqref{what-is-tex})。
Knuth 还设计了一个名为 \PlainTeX{} 的格式。所谓\strong{格式},是指对原始 \TeX{} 命令的封装,是
一种更高层次的抽象。它可以让用户在一定程度上摆脱排版中的繁琐细节,从而专注于文章写作。\LaTeX{} 就是
应用最为广泛的一种格式,它由 Leslie Lamport 发明,提供了 \TeX{} 的强大排版能力,但又足够易用,因此
深得人们的认可。现代比较流行的格式除 \PlainTeX{} 与 \LaTeX{} 之外,还有 \ConTeXt{}。

作为应用程序的 \TeX{} 也称之为\strong{排版引擎}。随着技术的发展,原始的 \TeX{} 程序逐渐不能满足
人们的需要。于是,新的引擎便被发明了出来,包括 \eTeX{}、\pdfTeX{}、\XeTeX{}、\LuaTeX{} 等。它们增加
了许多新的功能,比如 PDF 生成、Unicode 支持、OpenType 支持等。

\TeX{} 以及 \LaTeX{} 的使用涉及很多的应用程序;另一方面,社区又通过\strong{宏包}的形式引入了许多
额外功能。正如\faqref{install-latex} 中所介绍的那样,为了方便安装与使用,\strong{\TeX{} 发行版}将
程序、宏包以及相应的文档等打包在了一起,并以特定的结构(TDS,见\faqref{what-its-tds})进行管理。

排版引擎通常会与特定的格式组合起来,我们有时也简单地把它们叫做\strong{编译器}。前面所介绍的
\pdfTeX{}、\XeTeX{}、\LuaTeX{} 等默认都使用 \PlainTeX{} 格式,而使用 \LaTeX{} 格式的则相应地写为
\pdfLaTeX{}、\XeLaTeX{}、\LuaLaTeX{} 等。

在 \TeX{} 发行版中,实际上我们使用的都是一些可执行程序,具体见表~\ref{tab:engine-format-exe}。

\begin{table}[htb]
  \caption{各类引擎、格式及相应的可执行程序(命令)}
  \label{tab:engine-format-exe}
  \centering
  \begin{tabular}{*{5}{|c}|}
    \hline
    \diagbox{格式}{命令}{引擎} & \TeX & \pdfTeX & \XeTeX & \LuaTeX \\
    \hline
    \PlainTeX & |tex| & |etex|  / |pdftex|   & |xetex|   & |luatex|   \\
    \LaTeX    & ---   & |latex| / |pdflatex| & |xelatex| & |lualatex| \\
    \ConTeXt  & ---   & |texexec|            & !TODO!    & |context|  \\
    \hline
  \end{tabular}
\end{table}


\faq{怎样实现我想要的样式?}{get-my-style}

\LaTeX{} 推崇的原则称为“内容与格式分离”,如果您是刚入门的新手,请先专注于内容的编辑,而不是考虑如何
更改文章的样貌。一般而言,您所使用的模板都会预先实现所需的样式,而无需您自己想办法解决。

如果您对 \LaTeX{} 已经比较熟悉,并且需要自定义样式乃至制作新的模板,此问题自然也就不适用了。本文
之后的内容也许会对您有所帮助。


\faq{\LaTeX{} 生成的是什么文件?}{latex-generated-file}

现代 \TeX{} 引擎生成的都是 PDF 文件。在 Knuth 和 Lamport 的年代,PDF 还未被发明,因而传统上 \TeX{}
以 DVI 文件~\footnote{全称为 Device independent file format,注意与 DVI 数字视频接口(Digital
Visual Interface)区分。} 作为输出。曾经的一段时间里,PostScript(PS)格式也很流行,因而有 DVI
转换到 PS 的程序;当 PDF 流行起来之后,又出现了 PS 到 PDF 的转换程序。不过目前,这种复杂的工具链
都被淘汰了。


\faq{我应该用什么模板?}{how-to-use-template}

“模板”这个词往往带有歧义。我们说的模板,通常有以下几个意思:

\begin{itemize}
  \item 论文或者期刊文章的模板,如 \cls{thuthesis}、\cls{REVTeX} 等。它们已经定义好了样式,用户只
    需要填写内容。有些甚至提供了比较完善的示例,可以直接在此基础上修改、编辑。
  \item 表格模板、公式模板,这种并不很长的代码片段其实更应该叫做“示例”。我们鼓励新手学习、参考别人
    已有的代码,但并不鼓励直接照抄。一方面无助于提高自己水平,另一方面别人的代码片段可能与其他部分
    冲突,而且往往很难解决。
  \item 也有人用模板指代宏包或者文档类,不过这属于滥用术语,与别人交流还是用“某某宏包”、“某某文档
    类”比较合适。
  \item \cls{beamer} 文档类与 \LaTeX3 中的 \pkg{xtemplate} 宏包提供了更为抽象的模板机制,但这并不
    是给普通用户使用的。
\end{itemize}
